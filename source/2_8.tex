\subsection{Дифференцирование неявно заданной функции} \label{part:2.8}

Рассмотрим неявно заданную функцию $z = z(x,y)$ двух независимых переменных.\\

\textbf{Определение:} Уравнение \cref{eq:4.7} определяет функцию $z = z(x,y)$ как неявно заданную функцию двух переменных, если при подстановке этой функции в уравнение \cref{eq:4.7} оно становится тождеством:
\begin{equation} \label{eq:4.7}
	F(x,y,z) = 0
\end{equation}
\begin{equation} \label{eq:4.8}
	F(x,y,z(x,y)) \equiv 0
\end{equation}

Неявно заданную функцию из уравнения \cref{eq:4.7} можно найти при выполнении условий следующей теоремы.

\begin{tbox}{Теорема о существовании и дифференцируемости неявно заданной функции (без доказательства)}
	Пусть функция $F(x,y,z)$ и все ее частные производные $\frac{\partial F}{\partial x}$, $\frac{\partial F}{\partial y}$ и $\frac{\partial F}{\partial z}$ определены и непрерывны в некоторой окрестности точки $M_0(x_0,y_0, z_0)$, причем $F(M_0) = F(x_0, y_0, z_0) = 0$, а частная производной $\frac{\partial F(M_0)}{\partial z} \neq 0$.\\

	Тогда $\exists$ окрестность точки $M_0$ такая, что уравнение \cref{eq:4.7} определяет неявно заданную функцию $z = z(x,y)$, непрерывную и дифференцируемую в точке $(x_0, y_0)$, причем $z_0 = z(x_0, y_0)$.
\end{tbox}

Для нахождения частных производных подставляем в уравнение \cref{eq:4.7} функцию $z = z(x,y)$ и получаем тождество \cref{eq:4.8}, которое дифференцируем по независимым переменных $x$ и $y$.
\begin{gather*}
	F(x,y,z(x,y)) \equiv 0 \quad | \frac{\partial }{\partial x}\\
	\frac{\partial F}{\partial x} + \frac{\partial F}{\partial z} \cdot \frac{\partial z}{\partial x} = 0, \text{ откуда находим $\frac{\partial z}{\partial x}$}
\end{gather*}
\begin{equation} \label{eq:4.9}
	-\frac{\partial F}{\partial x} = \frac{\partial F}{\partial z} \cdot \frac{\partial z}{\partial x} \quad \Rightarrow \quad \boxed{\frac{\partial z}{\partial x} = -\frac{\frac{\partial F}{\partial x}}{\frac{\partial F}{\partial z}}}
\end{equation}
Аналогично, дифференцируя тождество \cref{eq:4.8} по $y$, найдем $\frac{\partial z}{\partial y}$:
\begin{equation} \label{eq:4.10}
	\frac{\partial F}{\partial y} + \frac{\partial F}{\partial z} \cdot \frac{\partial z}{\partial y} = 0 \quad \Rightarrow \quad \boxed{\frac{\partial z}{\partial y} = - \frac{\frac{\partial F}{\partial y}}{\frac{\partial F}{\partial z}}}
\end{equation}
Для нахождения частных производных неявно заданной функции формулы \cref{eq:4.9,eq:4.10} применять не будем а для каждого примера будем проводить данную процедуру.
\subsubsection*{Пример}
Найти $\frac{\partial z}{\partial x}$, $\frac{\partial z}{\partial y}$, если $z^3 - xz + y = 0$.
\begin{gather*}
	F(x,y,z) = z^3 - xz + y\\
	z^3(x,y) - xz(x,y) + y = 0 \quad \Big|\frac{\partial}{\partial x} \quad \Big|\frac{\partial}{\partial y}\\
	3z^2(x,y) z_x' - z(x,y) - x z_x' = 0 \quad (3z^2 - x) z_x' = 0\\
	z_x' = \frac{z}{3z^2 - x} \quad \text{($3z^2 - x \neq 0$, т.к. по условию теоремы $\frac{\partial F}{\partial z} = 3x^2 - x \neq 0$)}\\
	\begin{cases}
		3z^2(x,y) z_y' - xz_y' + 1 = 0\\
		(3z^2 - x)z_y' = -1
	\end{cases}
	\Rightarrow\quad z_y' = -\frac{1}{3z^2 - x}
\end{gather*}
Для нахождения частной производной 2-ого порядка дифференцируем найденные производные:
\begin{align*}
	z_x' &= \frac{z(x,y)}{3z^2(x,y) - x} \quad \left| \frac{\partial}{\partial x} \right. \\
	z_{xx}''
	&= \frac{z_x'(3z^2 - x) - z(6z \cdot z_x' - 1)}{(3z^2 - x)^2}
	= \frac{z - \dfrac{6z^3}{3z^2 - x} + z}{(3z^2 - x)^2}
	\quad \text{(подстановка $z_x'$)} \\
	&= \frac{2z - \dfrac{6z^3}{3z^2 - x}}{(3z^2 - x)^2}
	= \frac{2z(3z^2 - x) - 6z^3}{(3z^2 - x)^3} \\
	&= \frac{6z^3 - 2xz - 6z^3}{(3z^2 - x)^3}
	= -\frac{2xz}{(3z^2 - x)^3}\\
	z_y' &= -\frac{1}{3z^2(x,y) - x} \quad \left| \frac{\partial}{\partial y} \right. \quad \left| \frac{\partial}{\partial x} \right. \\
	z_{yy}''
	&= \frac{6z \cdot z_y'}{(3z^2 - x)^2}
	= \frac{6z \cdot \left(-\dfrac{1}{3z^2 - x}\right)}{(3z^2 - x)^2}
	= -\frac{6z}{(3z^2 - x)^3} \\
	z_{yx}''
	&= \frac{6z \cdot z_x' - 1}{(3z^2 - x)^2}
	= \frac{\dfrac{6z^2}{3z^2 - x} - 1}{(3z^2 - x)^2}
	= \frac{6z^2 - (3z^2 - x)}{(3z^2 - x)^3} \\
	&= \frac{3z^2 + x}{(3z^2 - x)^3}
\end{align*}

