\subsection{Условия монотонности функции в заданном направлении.}

\textbf{Определения: } Функция $y = f(\vec{x})$, $\vec{x} \in E \subset \mathbb{R}^k$, называется монотонно возрастающей в точке $\vec{x}_0 \in E \subset \mathbb{R}^k$ в направлении вектора $\vec{l} \subset \mathbb{R}^k$, $||\vec{l}|| = 1$, если существует окрестность точки $x_0$, такая, что для любых точке $\vec{x}_1$ и $\vec{x}_2$, $\forall \vec{x}_1$ и $\forall \vec{x}_2$ из этой окрестности, лежащиз на отрезке коллинеарным с вектором $\vec{l}$, так что $\vec{x}_1$ предшествует $\vec{x}_0$, а $\vec{x}_2$ следует за $\vec{x}_0$, выполняется неравенство.
\begin{equation} \label{eq:6.2}
	f(\vec{x}_1) < f(\vec{x}_0) \quad \text{ и } \quad f(\vec{x}_0) < f(\vec{x}_2)
\end{equation}

Если знак неравенства поменять на противоположный, то:
\begin{equation} \label{eq:6.3}
	f(\vec{x}_1) > f(\vec{x}_0) \quad \text{ и } \quad f(\vec{x}_0) > f(\vec{x}_2)
\end{equation}
получим определение функции монотонно убывающей в точке $\vec{x}_0$ в направлении $\vec{l}$.

\begin{tbox}{Теорема (Без доказательства)}
	Достаточное условие монотонности функции в заданном направлении.

	Пусть $y = f(\vec{x})$, дифференцируема в точке $\vec{x}_0 \in E \subset \mathbb{R}^k$,  $||\vec{l}|| = 1$, $\vec{l} \subset \mathbb{R}^k$ задает направление.

	\begin{itemize}
		\item Если $\displaystyle \frac{\partial \, F(\vec{x}_0)}{\partial l} > 0$, то функция $f(\vec{x})$ является монотонно возрастающей в точке $\vec{x}_0$ в направлении вектора $\vec{l}$.
		\item Если $\displaystyle\frac{\partial f(\vec{x}_0)}{\partial l} < 0$, то функция $f(\vec{x})$ является монотонно убывающей в точке $\vec{x}_0$ в направлении вектора $\vec{l}$.
		\item Если $\displaystyle\frac{\partial f(\vec{x}_0)}{\partial l} = 0$, то функция $f(\vec{x})$ не изменяется в направлении вектора $\vec{l}$, то есть равна $const$.
	\end{itemize}
\end{tbox}