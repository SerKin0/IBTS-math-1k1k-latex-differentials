\documentclass[9pt]{article}

\usepackage{bookstyle}
\usepackage{extsizes}
\usepackage{fancyhdr} % Пакет для настройки колонтитулов
\usepackage{graphicx} % Для работы с графикой (на случай, если используется в image/)

\pagestyle{fancy} % Устанавливаем стиль fancy для всех страниц
\fancyhf{} % Очищаем текущие колонтитулы

% Настраиваем rightmark для отображения названия подраздела
\renewcommand{\sectionmark}[1]{\markright{#1}} % Устанавливаем метку для раздела
\renewcommand{\subsectionmark}[1]{\markright{\thesubsection\ #1}} % Устанавливаем метку для подраздела

% Переопределяем стиль страницы для оглавления
\let\oldtoc\tableofcontents
\renewcommand{\tableofcontents}{\thispagestyle{fancy}\oldtoc}

\graphicspath{{./image/}} % Директория папки с фотографиями

\fancypagestyle{plain}{%
	\fancyhf{}
	\fancyhead[R]{Правый заголовок}
	\fancyfoot[C]{\small Страница \thepage}
	\fancyhead[R]{\small\itshape\rightmark}
}

\fancyhead[R]{\small\itshape\rightmark}
\fancyfoot[C]{\thepage} % Номер страницы внизу

\begin{document}
	\begin{titlepage}
	\newgeometry{
		left=10em,
		right=10em,
		top=2cm,
		bottom=2cm
	}
	\centering
	\MakeUppercase{Министерство науки и высшего образования российской федерации}

	\vspace{1cm}
	Федеральное государственное автономное образовательное учреждение высшего образования\\
	«Национальный исследовательский Нижегородский государственный университет им. Н.И. Лобачевского»

	\vspace{1cm}
	Радиофизический факультет

	\vspace{1.5cm}
	С. А. Скороходов

	\vspace{1cm}
	{\Huge\bfseries\MakeUppercase{Дифференциалы}}

	\vspace{1cm}
	Конспект по материалу 1 семестра\\
	Дисциплины -- Математический Анализ

	\vspace{1cm}
	Студента группы 417/0424С1ИБг1\\
	1 курса специалитета

	\vspace{1cm}
	Основная образовательная программа\\
	подготовки по направлению\\
	10.05.02 «Информационная безопасность\\
	телекоммуникационных систем»\\
	(направленность «Системы подвижной цифровой\\
	защищенной связи»)

	\vfill
	Нижний Новгород \\ Издательство "Невыспавшийся Студент" \\ 2025
\end{titlepage}
	\tableofcontents
	\section*{Предисловие}
\addcontentsline{toc}{section}{Предисловие}

Настоящий конспект представляет собой краткие записи по курсу «Математический анализ» по теме «Дифференциалы», оформленный с использованием \LaTeX. Он не претендует на статус полноценного учебного пособия и предназначен исключительно для личного использования при подготовке к занятиям и экзаменам.\\

Материал изложен с учётом программы курса, однако может содержать некоторые погрешности и упрощения. При использовании конспекта рекомендуется сверяться с дополнительными источниками и учебной литературой. Автор не несёт ответственности за результаты вашей сессии.\\

Распространение данного документа допускается только с личного согласия автора (Скороходов Сергей Александрович).\\

Выражаю особую признательность преподавателю дисциплины «Математический анализ» Семериковой Надежде Петровне за помощь в освоении курса и подготовке материалов для конспекта.\\

Для цитирования данного конспекта в работах, подготовленных в \LaTeX, рекомендуется использовать библиографическую запись следующего вида:
\begin{verbatim}
	@book{notediffserkin0,
		title = {Дифференциалы},
		author = {Скороходов, С.А.},
		publisher = {Издательство "Невыспавшийся Студент"},
		year = {2025},
		volume = {1},
		address = {Нижний Новгород},
		edition = {2-е изд., перераб.},
		language = {russian},
		url = {https://github.com/SerKin0/IBTS-math-1k1k-latex-differentials}
	}
\end{verbatim}
	\section{Основные понятия теории пределов и непрерывности функций многих переменных}
	\subsection{Понятие k-мерного Евклидова пространства} \label{sec:1.1}

Рассмотрим множество $\displaystyle \mathbb{R}^k = \mathbb{R \cdot R \cdot ... \cdot R}$ упорядоченных наборов действительных чисел длины $k$ $(x_1, x_2, ..., x_k)$, где $x_1 \in \mathbb R$, $x_2 \in \mathbb R$, ..., $x_k \in \mathbb R$.

Упорядоченный набор $(x_1, x_2, ..., x_k)$ называется \textbf{точкой} или \textbf{вектором на множестве} $\mathbb{R}^k$ и обозначается $\vec{x} = (x_1, x_2, ..., x_k)$, а действительные числа $x_1, x_2, ..., x_k$ называются координатными векторами или точками.\\

Пусть $k=2$, тогда множество $\mathbb{R}^k$ определяет плоскость \cref{fig:1.1.1.1}. Координаты любой точки на плоскости — это упорядоченная пара чисел $(x_1, x_2)$, эта пара чисел является координатами вектора, проведенного из начала координат в данную точку.

\begin{figure}[H]
	\centering
	\begin{minipage}{0.45\linewidth}
		\centering
		\includegraphics[width=0.9\linewidth]{image/screenshot001.png}
		\subcaption{При $k=2$}
		\label{fig:1.1.1.1}
	\end{minipage}
	\begin{minipage}{0.45\linewidth}
		\centering
		\includegraphics[width=0.9\linewidth]{image/screenshot006.png}
		\subcaption{При $k=3$}
		\label{fig:1.1.1.2}
	\end{minipage}
	\caption{Примеры $\mathbb{R}^k$ пространств}
	\label{fig:1.1.1}
\end{figure}

Аналогично, если $k=3$, то упорядоченный набор $(x_1, x_2, x_3)$ определяет точку или вектор в пространстве (\Cref{fig:1.1.1.2}).

Таким образом, элементами множества $\mathbb{R}^k$ являются \uline{векторы}. Над векторами вводятся следующие операции:
\begin{enumerate}
	\item \textbf{Сложение векторов}\\
	Если $\vec{x} = (x_1, x_2, ..., x_k) \in \mathbb{R}^k$ и $\vec{y} = (y_1, y_2, ..., y_k) \in \mathbb{R}^k$, то суммой векторов $(\vec{x} + \vec{y})$, будет являться сумма соответствующих координат:
	\begin{align} \label{1.1.1}
		(\vec{x} + \vec{y}) = (x_1 + y_1, x_2 + y_2, ..., x_k + y_k)
	\end{align}

	\item \textbf{Умножение вектора на скаляр}\\
	Если $\vec{x} = (x_1, x_2, ..., x_k)$ и $\alpha \in \mathbb R$ - действительное число, то $\alpha \vec{x} \in \mathbb{R}^k$ -- это вектор с координатами:
	\begin{align} \label{eq:1.1.2}
		(\alpha\vec{x}) = (\alpha \cdot x_1, \alpha \cdot x_2, ..., \alpha \cdot x_k)
	\end{align}

	\item \textbf{Скалярное произведение векторов}\\
	Если $\vec{x} = (x_1, x_2, ..., x_k) \in \mathbb{R}^k$ и $\vec{y} = (y_1, y_2, ..., y_k) \in \mathbb{R}^k$, тогда \textit{скалярным произведением векторов} называться скалярная величина, равная сумме произведений одноименных координат:
	\begin{align} \label{eq:1.1.3}
		(\vec{x} + \vec{y}) = x_1 y_1 + x_2 y_2 + ... + x_k y_k = \sum_{i = 1}^{k} x_i y_i
	\end{align}

	\item \textbf{Норма или длина вектора}\\
	Длина вектора $\vec{x} = (x_1, x_2, ..., x_k)$ вычисляется по формуле:
	\begin{align} \label{eq:1.1.4}
		||\vec{x}|| = \sqrt{x_1^2 + x_2^2 + ... + x_k^2} = \sqrt{\sum_{i=1}^{k} x_i^2}
	\end{align}

	\item \textbf{Расстояние между двумя точками или векторами}\\
	Если $\vec{x} = (x_1, x_2, ..., x_k) \in \mathbb{R}^k$ и $\vec{y} = (y_1, y_2, ..., y_k) \in \mathbb{R}^k$, то расстояние между точками $\rho(\vec{x}, \vec{y})$ определяется длиной вектора $(\vec{x} - \vec{y})$:
	\begin{multline} \label{eq:1.1.5}
		\rho(\vec{x}, \vec{y}) = ||\vec{x} - \vec{y}|| = \sqrt{(x_1 - y_1)^2 + (x_2 - y_2)^2 + ... + (x_k - y_k)^2} = \\ = \sqrt{\sum_{i=1}^{k} (x_i - y_i)^2}
	\end{multline}
\end{enumerate}

Если в множестве $\mathbb{R}^k$ введены рассмотренные выше операции с векторами, то оно называется \textbf{k-мерным Евклидовым пространством}.
	\subsection{Понятие функции многих переменных}

Начнем с определения функции одной переменной.

Если каждому числу $x$ из множества $\mathbb{E}$, которое является подмножеством действительных чисел $\mathbb{R}$, соответствует число $y$ из множества $Y$, также являющегося подмножеством $\mathbb{R}$ в соответствии с правилом $f$, то говорят, что на множестве $\mathbb{E}$ задана функция $y = f(x)$. Множество $\mathbb{E}$ называют областью определения функции, а $Y$ — множеством её значений.\\

Функция нескольких переменных определяется аналогично, только вместо одного числа используются несколько независимых переменных.

\begin{tbox*}{Определение функции k независимых переменных}
	Если каждому вектору $\vec{x} = (x_1, x_2, ..., x_k)$ из множества $\mathbb{E} \subset \mathbb{R}^k$ соответствует число $y$ из множества $Y \subset \mathbb{R}$ по правилу $f$, то на множестве $\mathbb{E}$ задана функция нескольких переменных, которую обозначают как $y = f(\vec{x})$ или $y = f(x_1, x_2, ..., x_k)$.

	Здесь $x_1, x_2, ..., x_k$ — независимые переменные (аргументы функции), а $y$ — зависимая переменная.

	Множество $\mathbb{E} \subset \mathbb{R}^k$ называют областью определения функции, а множество $Y \subset \mathbb{R}$ — её множеством значений.
\end{tbox*}

\subsubsection{Частные случаи функций многих переменных}
Рассмотрим функции двух и трех переменных. Для функции двух переменных:

\begin{itemize}
	\item Если $k=2$, то $y = f(x_1, x_2)$, что записывается как $z = f(x, y)$;
	\item Если $k=3$, то $y = f(x_1, x_2, x_3)$, что записывается как $w = f(x, y, z)$.
\end{itemize}

Особенно важна функция двух переменных $z = f(x, y)$, где $x$ и $y$ — независимые переменные. Область определения этой функции — множество точек $(x, y)$, принадлежащих некоторому подмножеству $\mathbb{E} \subset \mathbb{R}^2$. Зависимая переменная $z$ принимает значения из множества $Z \subset \mathbb{R}$, которое откладывается по вертикальной оси в пространстве XYZ.

По определению функции, каждой паре $(x, y) \in \mathbb{E}$ ставится в соответствие единственное значение $z$ по закону $f$. Это означает, что функция двух переменных имеет графическое представление в виде поверхности в пространстве. Эта поверхность состоит из всех значений функции во всех точках области определения $\mathbb{E}$.

\noindent
\begin{minipage}{\linewidth}
	\begin{minipage}{0.6\linewidth}
		\begin{tbox*}{Параболоид вращения (\Cref{fig:1.2.1.1})}
			\[\boxed{z = x^2 + y^2}\]
			О.О. $(x, y) \in \mathbb{R}$, $z \geqslant 0$ - множество значений
		\end{tbox*}
	\end{minipage}
	\begin{minipage}{0.35\linewidth}
		\begin{figure}[H]
			\centering
			\includegraphics[width=0.9\linewidth]{image/screenshot004.png}
			\caption{Параболоид}
			\label{fig:1.2.1.1}
		\end{figure}
	\end{minipage}
\end{minipage}
\begin{minipage}{\linewidth}
	\begin{minipage}{0.6\linewidth}
		\begin{tbox*}{Коническая поверхность (\Cref{fig:1.2.1.2})}
			\[\boxed{z^2 = x^2 + y^2}\]

			-- это неявно заданная функция. Выразим из уравнения $z$, $z = \pm \sqrt{x^2 + y^2}$ - получаем две явно заданные функции:
			\begin{enumerate}
				\item $\boxed{z = \sqrt{x^2 + y^2}}$, $(x, y) \in \mathbb{R}^2$, $z \geqslant 0$
				\item $\boxed{z = -\sqrt{x^2 + y^2}}$, $(x, y) \in \mathbb{R}^2$, $z \leqslant 0$
			\end{enumerate}
		\end{tbox*}
	\end{minipage}
	\begin{minipage}{0.35\linewidth}
		\begin{figure}[H]
			\centering
			\includegraphics[width=0.9\linewidth]{image/screenshot005.png}
			\caption{Коническая поверхность}
			\label{fig:1.2.1.2}
		\end{figure}
	\end{minipage}
\end{minipage}
\begin{minipage}{\linewidth}
	\begin{minipage}{0.6\linewidth}
		\begin{tbox*}{Сфера с цетром в начале (\Cref{fig:1.2.1.3})}
			\[\boxed{x^2 + y^2 + z^2 = R^2}\]

			Функция $z = \sqrt{R^2 - x^2 - y^2}$ задает верхнюю половину сферы. Здесь область определения $R^2 -x^2 - y^2 \geqslant 0 \Rightarrow x^2 + y^2 \leqslant R^2$ - круг радиуса $R$, а множество значений $0 \leqslant z \leqslant R$.\\

			Функция $z = -\sqrt{R^2 - x^2 - y^2}$ задает нижнюю половину сферы, область определения $x^2 + y^2 \leqslant R^2$, а множество значений $-R \leqslant z \leqslant 0$.\\

			\textbf{Замечание: } Функции, большего числа переменных, не имеют геометрического изображения.
		\end{tbox*}
	\end{minipage}
	\begin{minipage}{0.35\linewidth}
		\begin{figure}[H]
			\centering
			\includegraphics[width=0.9\linewidth]{image/screenshot007.png}
			\caption{Сфера с цетром в начале}
			\label{fig:1.2.1.3}
		\end{figure}
	\end{minipage}
\end{minipage}

	\subsection{Понятие предела функции многих переменных} \label{sec:1.3}
\begin{tbox}{Предел функции одной переменной}
	Вспомним определение предела для функции одной действительной переменной. Пусть \( y = f(x) \), где \( x \in \mathbb{E} \subset \mathbb{R} \). Точка \( x = a \) является предельной точкой множества \( \mathbb{E} \); она может как принадлежать \( \mathbb{E} \), так и не принадлежать ему (\( a \in \mathbb{E} \) или \( a \notin \mathbb{E} \)).

	\begin{multline} \label{eq:1.3.1}
		\lim_{x \to a} f(x) = A \Leftrightarrow
		(\forall \varepsilon > 0) \, (\exists \delta = \delta(\varepsilon) > 0) \,
		(\forall x \in \mathbb{E}, \, 0 < |x - a| < \delta) : \\
		|f(x) - A| < \varepsilon
	\end{multline}

	В определении предела неравенство \( 0 < |x - a| < \delta \) означает, что \( x \in (a - \delta, a + \delta) \) и \( x \neq a \). Геометрический смысл модуля \( |x - a| = \rho(x, a) \) — это расстояние между точками \( x \) и \( a \) на действительной оси, причём \( 0 < \rho(x, a) < \delta \).
\end{tbox}

\begin{tbox}{Предел функции многих переменных}
	Обобщим определение предела на случай функции многих переменных. Пусть \( y = f(\vec{x}) = f(x_1, x_2, \dots, x_k) \) определена на множестве \( \mathbb{E} \subset \mathbb{R}^k \). Точка \( \vec{a} = (a_1, a_2, \dots, a_k) \) является предельной для \( \mathbb{E} \) и может как принадлежать \( \mathbb{E} \), так и не принадлежать ему (\( \vec{a} \in \mathbb{E} \) или \( \vec{a} \notin \mathbb{E} \)).

	Расстояние между точками в \( \mathbb{R}^k \) было введено в \cref{sec:1.1}:
	\begin{align}
		\rho(\vec{x}, \vec{a}) = \|\vec{x} - \vec{a}\| =
		\sqrt{(x_1 - a_1)^2 + (x_2 - a_2)^2 + \dots + (x_k - a_k)^2}.
	\end{align}

	Предел функции многих переменных обозначается следующим образом:
	\begin{align}
		\lim_{\vec{x} \to \vec{a}} f(\vec{x}) = A \quad \text{или} \quad
		\lim_{\substack{x_1 \to a_1 \\ x_2 \to a_2 \\ \vdots \\ x_k \to a_k}}
		f(x_1, x_2, \dots, x_k) = A.
	\end{align}

	На языке «\(\varepsilon\)-\(\delta\)» определение аналогично \cref{eq:1.3.1}, но вместо чисел \( x \) и \( a \) используются векторы \( \vec{x} \) и \( \vec{a} \), а модуль \( |x - a| \) заменяется на норму \( \|\vec{x} - \vec{a}\| \):
	\begin{align}
		(\forall \varepsilon > 0) \, (\exists \delta = \delta(\varepsilon) > 0) \,
		(\forall \vec{x} \in \mathbb{E}, \, 0 < \|\vec{x} - \vec{a}\| < \delta) :
		|f(\vec{x}) - A| < \varepsilon
		\label{eq:2}
	\end{align}
\end{tbox}

\begin{tbox*}{Замечание о пределах}
	\textbf{Замечание.} Поскольку определение предела \eqref{eq:2} для функции многих переменных совпадает с определением для функции одной переменной, все теоремы о пределах, доказанные для случая одной переменной, переносятся на случай многих переменных.
\end{tbox*}

\begin{tbox}{Двойной предел}
	Рассмотрим предел функции двух переменных \( z = f(x, y) \), называемый двойным пределом. Пусть точка \( M(x, y) \in \mathbb{E} \subset \mathbb{R}^2 \) принадлежит области определения функции, а точка \( M_0(a, b) \) является предельной для \( \mathbb{E} \) (\( M_0 \in \mathbb{E} \) или \( M_0 \notin \mathbb{E} \)). Тогда:
	\begin{align}
		A = \lim_{M \to M_0} f(x, y) =
		\lim_{\substack{x \to a \\ y \to b}} f(x, y).
	\end{align}

	Расстояние между точками \( M \) и \( M_0 \) вычисляется по формуле:
	\[
	\rho(M, M_0) = \sqrt{(x - a)^2 + (y - b)^2} < \delta.
	\]

	На языке «\(\varepsilon\)-\(\delta\)» двойной предел записывается так:
	\begin{align}
		(\forall \varepsilon > 0) \, (\exists \delta = \delta(\varepsilon) > 0) \,
		(\forall (x, y) \in \mathbb{E}, \, 0 < \sqrt{(x - a)^2 + (y - b)^2} < \delta) : \\
		|f(x, y) - A| < \varepsilon
		\label{eq:3}
	\end{align}
\end{tbox}
\begin{figure}[t]
	\centering
	\includegraphics[width=0.7\linewidth]{image/screenshot008}
	\caption{Интервал $(a - \delta, a + \delta)$ с выколотой точкой}
	\label{fig:1.3.1}
\end{figure}
\begin{tbox}{Геометрический смысл двойного предела}
	Рассмотрим геометрический смысл неравенства:
	\begin{gather}
		0 < \sqrt{(x - a)^2 + (y - b)^2} < \delta = \delta(\varepsilon), \\
		0 < (x - a)^2 + (y - b)^2 < \delta^2(\varepsilon).
	\end{gather}

	Это задаёт круг радиуса \( \delta(\varepsilon) \) с выколотым центром в точке \( M_0(a, b) \). Такой круг называют \(\delta\)-окрестностью точки \( M_0 \) (\Cref{fig:1.3.2.1}).

	Для сравнения: в случае функции \( y = f(x) \) \(\delta\)-окрестность точки \( a \) — это интервал \( (a - \delta, a + \delta) \) с выколотой точкой \( a \) (\Cref{fig:1.3.1}).
\end{tbox}

\begin{tbox}{Независимость предела от пути}
	Из определения двойного предела следует, что если предел существует, то он не зависит от пути, по которому точка \( M \) приближается к \( M_0 \). Число возможных направлений бесконечно, в отличие от функции одной переменной, где таких направлений всего два (слева и справа от точки \( a \)).
\end{tbox}

\begin{figure}[H]
	\centering
	\begin{minipage}{0.45\linewidth}
		\centering
		\includegraphics[width=0.9\linewidth]{image/screenshot009.png}
		\caption{ }
		\label{fig:1.3.2.1}
	\end{minipage}
	\begin{minipage}{0.45\linewidth}
		\centering
		\includegraphics[width=0.9\linewidth]{image/screenshot010.png}
		\caption{ }
		\label{fig:1.3.2.2}
	\end{minipage}
\end{figure}

\subsubsection{Примеры решения двойных пределов}
\begin{enumerate}
	\item Вместо $x$ и $y$ подставляем предельные значения:
	\begin{align*}
		\lim_{\tiny{\begin{array}[b]{c} x \to 1\\y \to 2 \end{array}}} \frac{x \cdot y}{x^2 + y^2} =
		\frac{1 \cdot 2}{1^2 + 2^2} =
		\frac{2}{5}
	\end{align*}

	\item По теореме и произведении бесконечно малых на ограниченную:
	\begin{align*}
		\lim_{\tiny{\begin{array}[b]{c} x \to 0\\y \to 0 \end{array}}} (x + y \cdot \sin \frac{1}{x}) =
		\lim_{x \to 0} x + \lim_{\tiny{\begin{array}[b]{c} x \to 0\\y \to 0 \end{array}}} y \cdot \sin \frac{1}{x} =
		0
	\end{align*}

	\item Используя первый замечательный предел:
	\begin{multline*}
		\lim_{\tiny{\begin{array}[b]{c} x \to \infty \\y \to 2 \end{array}}} \left(x \cdot \sin \frac{1}{xy}\right) \, [\infty \cdot 0] =
		\lim_{\tiny{\begin{array}[b]{c} x \to \infty \\y \to 2 \end{array}}} \left(\frac{\sin \frac{1}{xy}}{\frac{1}{x}}\right) \left[\frac{0}{0}\right] =\\=
		\lim_{\tiny{\begin{array}[b]{c} x \to \infty \\y \to 2 \end{array}}} \left(\frac{\sin \frac{1}{xy}}{\frac{1}{xy} \cdot y}\right) =
		\lim_{y \to 2} \frac{1}{y} = \frac{1}{2}
	\end{multline*}

	\item Покажем что предел не существует. Для этого выберем окрестность предельной точки $M_0(0,0)$ и предположим, что точка $M(x, y) \to M_0(0, 0)$ по различным путям (выше уже было сказано, что число таких направлений бесконечно). Для простоты выберем две прямые: $y = x$ и $y = -x$
	\begin{align*}
		\lim_{\tiny{\begin{array}{c} x \to 0 \\[-3pt] y \to 0 \end{array}}} \frac{xy}{x^2 + y^2}
		= \left| \begin{array}{l} y = x \\[-3pt] x \to 0 \\[-3pt] y \to 0 \end{array} \right| = \lim_{x \to 0} \frac{x^2}{x^2 + x^2} = \lim_{x \to 0} \frac{x^2}{2x^2} = \frac{1}{2}
	\end{align*}
	\begin{align*}
		\lim_{\tiny{\begin{array}{c} x \to 0 \\[-3pt] y \to 0 \end{array}}} \frac{xy}{x^2 + y^2}
		= \left| \begin{array}{l} y = -x \\[-3pt] x \to 0 \\[-3pt] y \to 0 \end{array} \right| = \lim_{x \to 0} \frac{x \cdot (-x)}{x^2 + (-x)^2} = \lim_{x \to 0} \frac{x^2}{2x^2} = -\frac{1}{2}
	\end{align*}
	Таким образом, рассмотрели 2 частных предела, они не равны между собой, следовательно двойной предел не существует.
\end{enumerate}
	\subsection{Понятие непрерывности функции многих переменных}

\cref{eq:1.1.2}

	\newpage
	\section{Дифференцирование функций многих переменных}
	\subsection{Определение частных производных и их геометрический смысл} \label{sec:2.1}

\begin{tbox}{Функция одной переменной}
	Рассмотрим функцию \( y = f(x) \), где \( x \in E \subset \mathbb{R} \) и \( y \in E \).
	Запишем определение производной в точке \( x_0 \). Для этого зададим приращение аргумента \( \Delta x = x - x_0 \) (или \( x = x_0 + \Delta x \)) и вычислим приращение функции:
	\[
	\Delta f(x_0) = f(x_0 + \Delta x) - f(x_0).
	\]

	Производной функции в точке называется предел отношения приращения функции к приращению аргумента при \( \Delta x \to 0 \):
	\begin{align}
		\lim_{\Delta x \to 0} \frac{\Delta f(x_0)}{\Delta x} = f'(x_0) = \frac{dy}{dx}\bigg|_{x = x_0} = \frac{df(x_0)}{dx}.
		\label{eq:15}
	\end{align}
\end{tbox}

\begin{tbox}{Функция многих переменных}
	Теперь обобщим определение (\ref{eq:15}) на случай функции многих переменных \( y = f(\vec{x}) = f(x_1, x_2, \dots, x_k) \). Поскольку дифференцирование проводится по одной переменной, зададим приращение в точке \( \vec{x}_0 \) только по переменной \( x_i \). Вектор приращений имеет вид:
	\[
	\Delta \vec{x} = (0, \dots, 0, \Delta x_i, 0, \dots, 0),
	\]

	тогда:
	\[
	\vec{x} = \vec{x}_0 + \Delta \vec{x} = \left(x_0^{(1)}, \dots, x_0^{(i-1)}, x_0^{(i)} + \Delta x_i, x_0^{(i+1)}, \dots, x_0^{(k)}\right).
	\]

	Соответствующее частное приращение функции:
	\begin{align}
		\Delta_i f(\vec{x}_0) = f\left(x_{0_1}, \dots, x_{0_{i-1}}, x_{0_i} + \Delta x_i, x_{0_{i+1}}, \dots, x_{0_k}\right) - f\left(x_{0_1}, \dots, x_{0_k}\right).
		\label{eq:16}
	\end{align}
\end{tbox}

\begin{tbox}{Частная производная}
	По аналогии с \cref{eq:15} определим частную производную как предел:
	\begin{align}
		\exists \lim_{\Delta x_i \to 0} \frac{\Delta_i f(\vec{x}_0)}{\Delta x_i}.
		\label{eq:17}
	\end{align}

	Если этот предел существует, то он называется частной производной функции \( y = f(\vec{x}) \) в точке \( \vec{x}_0 \) по переменной \( x_i \) и обозначается:
	\begin{align}
		f_{x_i}^{\prime}(\vec{x}_0) = \frac{\partial f(\vec{x}_0)}{\partial x_i}.
		\label{eq:18}
	\end{align}
	\textbf{Замечание}: Запись \( \frac{df(\vec{x}_0)}{dx_i} \) не используется для частных производных.
\end{tbox}

\begin{tbox}{Частные производные первого порядка}
	Поскольку производная вычисляется по одной из \( k \) независимых переменных, функция \( y = f(x_1, x_2, \dots, x_k) \) имеет \( k \) частных производных первого порядка:
	\begin{align}
		\frac{\partial f}{\partial x_1}, \frac{\partial f}{\partial x_2}, \dots, \frac{\partial f}{\partial x_k}.
		\label{eq:19}
	\end{align}
\end{tbox}

\begin{tbox}{Функция двух переменных}
	Рассмотрим частный случай \( z = f(x, y) \). В точке \( M_0(x_0, y_0) \) зададим приращение \( \Delta x \) по переменной \( x \). Частное приращение:
	\begin{align}
		\Delta_x z(M_0) = f(x_0 + \Delta x, y_0) - f(x_0, y_0).
		\label{eq:20}
	\end{align}

	Частная производная по \( x \):
	\begin{align}
		\frac{\partial z(M_0)}{\partial x} = \lim_{\Delta x \to 0} \frac{\Delta_x z(M_0)}{\Delta x}.
		\label{eq:21}
	\end{align}

	Аналогично для приращения \( \Delta y \) по \( y \):
	\begin{align}
		\frac{\partial z(M_0)}{\partial y} = \lim_{\Delta y \to 0} \frac{f(x_0, y_0 + \Delta y) - f(x_0, y_0)}{\Delta y}.
		\label{eq:22}
	\end{align}
\end{tbox}

\begin{tbox*}{Правила вычисления}
	При вычислении \( \frac{\partial z}{\partial x} \) переменная \( y \) считается константой, и наоборот. Используются стандартные правила дифференцирования:
	\begin{itemize}
		\item Производная константы равна нулю.
		\item Константа выносится за знак производной.
	\end{itemize}
\end{tbox*}

\begin{tbox}{Геометрический смысл}
	Функция \( z = f(x, y) \) задаёт поверхность в \( \mathbb{R}^3 \). Точке \( M_0(x_0, y_0) \) соответствует точка \( N(x_0, y_0, z_0) \) на поверхности.\\

	\textbf{Производная по x (\Cref{fig:2.1.1.1})}. При \( y = y_0 \) получаем сечение поверхности плоскостью, параллельной \( XOZ \). Частная производная \( \frac{\partial z}{\partial x} \) равна тангенсу угла \( \alpha \) наклона касательной к этому сечению:
	\[
	\tg \alpha = \frac{\partial z(M_0)}{\partial x}.
	\]

	\textbf{Производная по y (\Cref{fig:2.1.1.2})}. Аналогично, при \( x = x_0 \):
	\[
	\tg \beta = \frac{\partial z(M_0)}{\partial y}.
	\]
\end{tbox}

\begin{figure}[H]
	\centering
	\begin{minipage}{0.45\linewidth}
		\centering
		\includegraphics[width=0.9\linewidth]{screenshot011}
		\caption{Геометрический смысл \( \frac{\partial z}{\partial x} \).}
		\label{fig:2.1.1.1}
	\end{minipage}
	\begin{minipage}{0.45\linewidth}
		\centering
		\includegraphics[width=0.9\linewidth]{screenshot012}
		\caption{Геометрический смысл \( \frac{\partial z}{\partial y} \).}
		\label{fig:2.1.1.2}
	\end{minipage}
\end{figure}
	\subsection{Дифференцируемые функции} \label{sec:2.2}

\begin{tbox}{Определение дифференцируемой функции $y = f(x)$ в точке $x_0$}
	Функция $y = f(x)$ называется \textbf{дифференцируемой в точке} \( x_0 \), если её приращение в этой точке представимо в виде:
	\begin{align} \label{eq:2.2.1}
		\Delta f(x_0) = f(x_0 + \Delta x) - f(x_0) = A \cdot \Delta x + \alpha(\Delta x) \to 0
	\end{align}
	где $A = const$, $\alpha(\Delta x)$ - бесконечно малая при \(\Delta x \to 0\), т.е. $\alpha(\Delta x) \cdot \Delta x$ -- величина бесконечно малая, более высокого порядка малости, чем $\Delta x$.
\end{tbox}

\begin{tbox}{Необходимое и достаточное условие дифференцируемости функции одной переменной}
	Существование конечной производной в точке $x_0$.\\

	Было доказано, что $A = f^{\prime}(x_0)$ и определение дифференцируемой функции можно представить следующим образом:
	\begin{align} \label{eq:2.2.2}
		\Delta f(x_0) = f^{\prime}(x_0) \cdot \Delta x + \alpha(\Delta x) \cdot \Delta x \qquad \alpha (\Delta x) \cdot \Delta x = O(\Delta x)
	\end{align}
\end{tbox}

\begin{tbox}{Определение функции многих переменных $y = f(\vec{x})$}
	Функция $\boxed{y = f(\vec{x})} = f(x_1, x_2, ..., x_k)$, где $\vec{x} \in \mathbb{E} \subset \mathbb{R}^k$  называется дифференцируемой в точке $\boxed{\vec{x}_0} = (x_{\text{$0_1$}}, x_{\text{$0_2$}}, ..., x_{\text{$0_k$}}) \in \mathbb{E} \subset \mathbb{R}^k$, если ее полное приращение:
	\begin{multline*}
		\Delta f(\vec{x}_0) = f(\vec{x}_0 + \Delta \vec{x}) - f(\vec{x}_0) =\\= f(x_{\text{$0_1$}} + \Delta x_1, x_{\text{$0_2$}} + \Delta x_2, ..., x_{\text{$0_k$}} + \Delta x_k) - f(x_{\text{$0_1$}}, x_{\text{$0_2$}}, ..., x_{\text{$0_k$}})
	\end{multline*}
	Также можно представить в виде:
	\begin{align} \label{eq:2.2.3}
		\boxed{\Delta f(\vec{x}_0) = \vec{A} \cdot \Delta \vec{x} + \vec{\alpha}(\Delta \vec{x}) \cdot \Delta \vec{x}}
	\end{align}
	\begin{itemize}
		\item $\vec{A} = (A_1, A_2, ..., A_k)$ -- постоянный вектор;
		\item $\Delta \vec{x} = (\Delta x_1, \Delta x_2, \dots, \Delta x_k)$ -- вектор приращений;
		\item $\vec{\alpha}(\Delta \vec{x}) = (\alpha_1(\Delta \vec{x}), \alpha_2(\Delta \vec{x}), ..., \alpha_k(\Delta \vec{x}))$, причем $\alpha_1(\Delta \vec{x}) \to \vec{0}$, при $\Delta \vec{x} \to \vec{0}$. ($\vec{0}$ -- ноль вектор);
	\end{itemize}
\end{tbox}
Распишем в определении \ref{eq:2.2.3} скалярное произведение:
\begin{gather*}
	\vec{A} \cdot \Delta \vec{x} = A_1 \Delta x_1 + A_2 \Delta x_2 + \dots + A_k \Delta x_k = \sum_{i=1}^k A_i \Delta x_i \\
	\vec{\alpha}(\Delta \vec{x}) \cdot \Delta \vec{x} = \alpha_1 (\Delta \vec{x}) \cdot \Delta x_1 + \alpha_2 (\Delta \vec{x}) \cdot \Delta x_2 + ... + \alpha_k (\Delta \vec{x}) \cdot \Delta x_k = \sum_{i = 1}^{k} \alpha_i (\Delta x) \Delta x_i
\end{gather*}

Тогда получаем определение дифференцируемой функции в \textbf{координатной форме}:
\begin{align} \label{eq:2.2.4}
	\boxed{\Delta f(\vec{x}_0) = \sum_{i=1}^k A_i \cdot \Delta x_i + \sum_{i=1}^k \alpha_i (\Delta \vec{x}) \cdot x_i}
\end{align}

\begin{tbox}{Необходимые условия дифференцируемости}
	Если функция \( y = f(\vec{x}) \) дифференцируема в точке \( \vec{x}_0 \), то она имеет в этой точке частные производные.\\

	\textbf{Доказательство:} Запишем определение дифференцируемой функции в координатной форме \ref{eq:2.2.3}.
	\begin{align*}
		\Delta f(\vec{x}_0) = \sum_{i=1}^k A_i \Delta x_i + \sum_{i=1}^{k} \alpha_i (\Delta \vec{x}) \cdot \vec{x}_i
	\end{align*}
	Зададим вектор приращений $\Delta \vec{x}$ в виде: $\Delta \vec{x} = (0, ..., 0, \Delta x_i, 0, ..., 0)$.
	Тогда полное приращение в записанной формуле будет совпадение с частным приращением в точке $\vec{x}_0$ по переменой $x_i$ и в сумме останется только два слагаемых:
	\begin{gather*}
		\Delta f(\vec{x}_0) = \Delta_i f(\vec{x}_0) = A_i \cdot \Delta x_i + \alpha_i(\Delta \vec{x}) \cdot x_i \qquad \text{(поделим обе части на $\Delta x_i$)}\\
		\frac{\Delta_i f(\vec{x}_0)}{\Delta x_i} = \frac{A_i \cdot \Delta x_i + \alpha_i (\Delta \vec{x}) \cdot \Delta x_i}{\Delta x_i} = A_i + \alpha_i(\Delta \vec{x}) \quad |_{\lim_{\Delta x_i \to 0}}\\
		\lim_{\Delta x_i \to 0} \frac{\Delta_i f(\vec{x}_0)}{\Delta x_i} = \lim_{\Delta x_i \to 0} \left(A_i + \alpha_i(\Delta \vec{x})\right) = A_i + \lim_{\Delta x_i \to 0} \alpha_i(\Delta \vec{x}) = A_i
	\end{gather*}
	С другой стороны: $\displaystyle A_i = \lim_{\Delta x_i \to 0} \frac{\Delta_i f(\vec{x}_0)}{\Delta x_i} = \frac{\partial f(\vec{x}_0)}{\partial x_i} \quad \Rightarrow \quad \boxed{A_i = \frac{\partial f(\vec{x}_0)}{\partial x_i}} \quad (i = 1,2,...,k)$.\\

	Тогда получается, что координаты постоянного вектора $\vec{A}$ равны частным производным функции $f(\vec{x})$ в точке $\vec{x}_0$ по всем независимым переменным.
	\begin{align}
		\vec{A} = (\frac{\partial f(\vec{x}_0)}{\partial x_1}, \frac{\partial f(\vec{x}_0)}{\partial x_2}, ..., \frac{\partial f(\vec{x}_0)}{\partial x_k})
	\end{align}

	Вектор, координатами которого являются частные производные, называется градиентом функции в точке $\vec{x}_0$ и обозначается $\grad f(\vec{x}_0)$.

	\begin{align*}
		\vec{A} = \grad f(\vec{x}_0) = \left(\frac{\partial f(\vec{x}_0)}{\partial x_1}, \frac{\partial f(\vec{x}_0)}{\partial x_2}, ..., \frac{\partial f(\vec{x}_0)}{\partial x_k}\right)
	\end{align*}
	Тогда из определений \cref{eq:2.2.2,eq:2.2.3} получаем еще одно определение дифференцируемой функции.
	\begin{align}
		\Delta f(\vec{x}_0) = \grad f(\vec{x}_0) \cdot \Delta \vec{x} + \vec{\alpha} (\Delta \vec{x}) \cdot \Delta \vec{x}
		\label{eq:2.2.5}
	\end{align}
	\vspace{-2em}
	\begin{align}
		\Delta f(\vec{x}_0) = \sum_{i=1}^{k} \frac{\partial f(\vec{x}_0)}{\partial x_i} \Delta x_i + \sum_{i=1}^{k} \alpha_i (\Delta \vec{x}) \cdot \Delta x_i
		\label{eq:2.2.6}
	\end{align}
\end{tbox}

\begin{tbox}{Достаточное условие дифференцируемости}
	Если функция $y = f(\vec{x})$ имеет частные производные по всех переменным в окрестности точки $\vec{x}_0 \in \mathbb{E} \subset \mathbb{R}^k$, причем частные производные:
	\begin{align*}
		\left(\frac{\partial f(\vec{x}_0)}{\partial x_1}, \frac{\partial f(\vec{x}_0)}{\partial x_2}, ..., \frac{\partial f(\vec{x}_0)}{\partial x_k}\right) \text{-- непрерывна в точке $\vec{x}_0$,}
	\end{align*}
	то функция $y = f(\vec{x})$ дифференцируема в точке $\vec{x}_0$. (Без доказательства)
\end{tbox}

Рассмотрим частный случай $k = 2$ и из определений \cref{eq:2.2.5,eq:2.2.6} запишем определение дифференцируемой функции 2-х переменных в точке $\vec{x}_0 = (x_{\text{$0_1$}}, x_{\text{$0_2$}})$.
\[y = f(x_1, x_2), \qquad \grad f(\vec{x}_0) = \left(\frac{\partial f(\vec{x}_0)}{\partial x_1}, \frac{\partial f(\vec{x}_0)}{\partial x_2}\right)\]
\[\Delta \vec{x} = (\Delta x_1, \Delta x_2), \qquad \vec{\alpha}(\Delta \vec{x}) = \left(\alpha_1(\Delta x_1, \Delta x_2), \alpha_2(\Delta x_1, \Delta x_2)\right)\]
\begin{multline*}
	\Delta f(\vec{x}_0) = \grad f(\vec{x}_0) \cdot \Delta \vec{x} + \vec{\alpha}(\Delta \vec{x}) \cdot \Delta \vec{x} =\\= \frac{\partial f(\vec{x}_0)}{\partial x_1} \Delta x_1 + \frac{\partial f(\vec{x}_0)}{\partial x_2} \Delta x_2 + \alpha_1(\Delta x_1 \Delta x_2) \Delta x_1 + \alpha_2 (\Delta x_1 \Delta x_2) \Delta x_2
\end{multline*}

Рассмотрим функцию $z = f(x,y)$, тогда в полученном определении заменим $x_1$ на $x$, а $x_2$ на $y$, $\Delta \vec{x} = (\Delta x, \Delta y)$, $\vec{\alpha}(\Delta \vec{x}) = (\alpha_1(\Delta x, \Delta y), \alpha_2(\Delta x, \Delta y))$.\\

Получаем определение \textbf{дифференцируемой функции для двух переменных}.
\begin{align} \label{eq:2.2.7}
	\boxed{\Delta z(M_0) = \frac{\partial f(M_0)}{\partial x} \Delta x + \frac{\partial f(M_0)}{\partial y} \Delta y + \alpha_1(\Delta x, \Delta y) \Delta x + \alpha_2 (\Delta x, \Delta y) \Delta y}
\end{align}

\begin{tbox}{Теорема о связи между непрерывностью и дифференцированностью функции многих переменных}
	Если функции $y = f(\vec{x})$ дифференцируема в точке $\vec{x}_0 \in \mathbb{E} \subset \mathbb{R}^k$, то она непрерывна в этой точке.\\

	\textbf{Доказательство:}  Запишем определение дифференцируемой функции в форме \cref{eq:2.2.3}:
	\[\Delta f(\vec{x}_0) = \vec{A} \cdot \Delta \vec{x} + \vec{\alpha}(\Delta \vec{x}) \cdot \Delta \vec{x}\], здесь $\vec{\alpha}(\Delta \vec{x}) \to \vec{0}$, при $\Delta \vec{x} \to \vec{0}$. Перейдем к пределу при $\Delta \vec{x} \to \vec{0}$.
	\begin{align*}
		\lim_{\Delta \vec{x} \to \vec{0}} \Delta f(\vec{x}_0) = \lim_{\Delta \vec{x} \to \vec{0}} \left(\vec{A} \cdot \Delta \vec{x} + \vec{\alpha}(\Delta \vec{x}) \cdot \Delta \vec{x}\right) = 0
	\end{align*}
	т.е. малым приращениям аргументов ($\Delta \vec{x} = (\Delta x_1, \Delta x_2, ..., \Delta x_k) \to \vec{0}$) соответствует малое полное приращение функции. Это означает, что функция $y = f(\vec{x})$ непрерывна в точке $\vec{x}_0$ по совокупности переменных.\\

	\textbf{Замечание:} Обратное утверждение не всегда верно, по аналогии с функцией $y = |x|$ в точке $x = 0$. Функция непрерывна при $x=0$, но в точке не имеет производной.
\end{tbox}
	\subsection{Дифференциальная функция многих переменных} \label{sec:1.4}

Если функция $y = f(x)$ дифференцируема в точке $x_0$, что ее приращение представимо в виде:
\begin{align*}
	\Delta f(x_0) = A \cdot \Delta x + \alpha(\Delta x) \cdot \Delta x = A \cdot \Delta x + O(\Delta x) \quad \text{\textit{, т.к.} $\alpha(\Delta x) \to 0$ \textit{при} $\Delta x \to 0$}
\end{align*}

Приращение функции состоит из двух частей, $A \cdot \Delta x$ -- линейной относительно $\Delta x$ и величине более высокого порядка малости, чем $\Delta x$. Главная линейная часть называется дифференциалом функции и обозначается $d y = A \cdot \Delta x$, но $A = f^{\prime}(x_0)$, а $\Delta x = d x$, тогда $\boxed{d y = f^{\prime}(x_0) \, d x}$.

Рассмотрим функцию многих переменных $y = f(\vec{x})$, $\vec{x} \in E \subset \mathbb{R}^k$. Запишем определение дифференцируемой функции в точке $x_0$ в виде \cref{eq:2.2.5,eq:2.2.6}.
\begin{gather*}
	\Delta f(\vec{x}_0) = \boxed{\grad f(\vec{x}_0) \cdot \Delta \vec{x}} + \vec{\alpha}(\Delta \vec{x}) \cdot \Delta \vec{x}\\
	\Delta f(\vec{x}_0) = \boxed{\sum_{i=1}^{k} \frac{\partial f(\vec{x}_0)}{\partial x_i} \Delta x_i} + \sum_{i=1}^{k} \alpha_i(\Delta \vec{x}) \cdot \Delta x_i
\end{gather*}

Выделенные части определения представляет линейную относительно $\Delta \vec{x}$ часть приращения функции, которая называется \textbf{дифференциалом}.
\begin{multline*}
	d y = \grad f(\vec{x}_0) \cdot \Delta \vec{x} = \sum_{i=1}^{k} \frac{\partial f(\vec{x}_0)}{\partial x_i} \Delta x_i =\\= \frac{\partial f(\vec{x}_0)}{\partial x_1} \Delta x_1 + \frac{\partial f(\vec{x}_0)}{\partial x_2} \Delta x_2 + ... + \frac{\partial f(\vec{x}_0)}{\partial x_k} \Delta x_k
\end{multline*}

\textbf{Приращения} независимых переменных обозначим через \textbf{дифференциал} независимых переменных: $\Delta x_1 = d x_1$, $\Delta x_2 = d x_2$, ..., $\Delta x_k = d x_k$.

Тогда дифференциалом функции многих переменных будет равен:
\begin{align*}
	\boxed{d y = \frac{\partial f(\vec{x}_0)}{\partial x_1} d x_1 + \frac{\partial f(\vec{x}_0)}{\partial x_2} d x_2 + ... + \frac{\partial f(\vec{x}_0)}{\partial x_k} d x_k = \sum_{i=1}^{k} \frac{\partial f(\vec{x}_0)}{\partial x_i} d x_i}
\end{align*}

Выражения вида $\frac{\partial f(\vec{x}_0)}{\partial x_i} d x_i$ называются \textbf{частными дифференциальными} и обозначаются:
\begin{align*}
	d_i f(\vec{x}_0) = \frac{\partial f(\vec{x}_0)}{\partial x_i} d x_i \quad (i = 1,...,k)
\end{align*}

Тогда полный дифференциал функции $d y$ равен сумме частных дифференциалов по всех независимым переменным.
\begin{align*}
	d y = \sum_{i=1}^{k} d_i \, f(x_0)
\end{align*}

\begin{center}
	\textbf{Примеры}
\end{center}

\begin{enumerate}
	\item \(z = f(x, y) \quad \Rightarrow \quad d_x \, z = f_x^{\prime} d x, \qquad d_y \, z = f^{\prime}_y d y\)

	Тогда для функции двух переменных дифференциал равен: \\
	\[d z = f^{\prime}_x \, d x + f^{\prime}_y \, d y = \frac{\partial f}{\partial x} d x + \frac{\partial f}{\partial y} d y
	\]
	\item \(u = f(x, y, z) \quad \Rightarrow \quad d_x \, u = \frac{\partial f}{\partial x} d x, \quad d_y \, u = \frac{\partial f}{\partial y} d y, \quad d_z \, u = \frac{\partial f}{\partial z} d z\)

	Для функции трех переменных дифференциал вычисляется по формуле: \\
	\[d u = f^{\prime}_x \, d x + f^{\prime}_y \, d y + f^{\prime}_z \, d z = \frac{\partial f}{\partial x} d x + \frac{\partial f}{\partial y} d y + \frac{\partial f}{\partial z} d z \]
\end{enumerate}


	\subsection{Производная сложной функции} \label{sec:1.5}
Рассмотрим на примере функции двух переменных. Пусть задана функция $z = f(u, v)$, а ее аргументы являются функциями переменных $x$ и $y$: $u=u(x,y)$ и $v=v(x,y)$. Тогда получаем сложную функцию $z = f(u(x,y), v(x,y))$ независимых переменных $x$ и $y$. Функции $u = u(x,y)$ и $v = v(x,y)$ независимых промежуточными аргументами.

В дальнейшем будем рассматривать функцию двух промежуточных и двух независимых переменных.

\begin{tbox*}{Теорема}
	Если функция $z = f(u, v)$ дифференцируема в точке $(u_0, v_0) \in D \subset \mathbb R^2$, а функции $u = u(x,y)$, $v = v(x,y)$ дифференцируемы в точке $(x_0, y_0) \in E \subset \mathbb R^2$, причем $u_0 = u(x_0, y_0)$ и $v_0 = v(x_0, y_0)$.

	Тогда сложная функция $z = f(u(x,y), v(x,y))$ дифференцируема в точке $(x_0, y_0)$ и ее частные производные в этой точке вычисляются по формулам:

	\begin{gather*}
		\frac{\partial z(x_0, y_0)}{\partial x}  = \frac{\partial f(u_0, v_0)}{\partial u} \times \frac{\partial u(x_0, y_0)}{\partial x} + \frac{\partial f(u_0, v_0)}{\partial v} \times \frac{\partial v(x_0, y_0)}{\partial x}\\
		\frac{\partial z(x_0, y_0)}{\partial y} = \frac{\partial f(u_0, v_0)}{\partial u} \times \frac{\partial u(x_0, y_0)}{\partial y} + \frac{\partial f(u_0, v_0)}{\partial v} \times \frac{\partial v(x_0, y_0)}{\partial y}
	\end{gather*}

	\textbf{Доказательство:} Воспользуемся определением дифференцируемой функции двух переменных \cref{eq:2.2.7}.
	\begin{align} \label{eq:2.4.1}
		\Delta z(M_0) = \frac{\partial f(M_0)}{\partial x} \Delta x + \frac{\partial f(M_0)}{\partial y} \Delta y + \alpha_1 (\Delta x, \Delta y) \Delta x + \alpha_2 (\Delta x, \Delta y) \Delta y
	\end{align}
	где $a_1(\Delta x, \Delta y) \to 0$ и $\alpha_2(\Delta x, \Delta y) \to 0$, при $\Delta x \to 0$ и $\Delta y \to 0$. \\

	Так как функция $z = f(u, v)$ дифференцируема в точке $(u_0, v_0)$, то ее полное приращение запишем в виде:
	\begin{align} \label{eq:2.4.2}
		\Delta z (u_0, v_0) = \frac{\partial f(u_0, v_0)}{\partial u} \Delta u + \frac{\partial f(u_0, v_0)}{\partial v} \Delta v + \alpha(\Delta u, \Delta v) \Delta u + \beta (\Delta u, \Delta v) \Delta v
	\end{align}
	где $\alpha(\Delta u, \Delta v) \to 0$ и $\beta(\Delta u, \Delta v) \to 0$, при $\Delta y \to 0$ и $\Delta v \to 0$. Т.к. функции $u = u(x,y)$ и $v = v(x,y)$ дифференцируемы в точке $(x_0, y_0)$, то их полные приращения имеют вид:
	\begin{align} \label{eq:2.4.3}
		\Delta u(x_0, y_0) = \frac{\partial u(x_0, y_0)}{\partial x} \Delta x + \frac{\partial u(x_0, y_0)}{\partial y} \Delta y + \alpha_1(\Delta x, \Delta y) \Delta x + \beta_1 (\Delta x, \Delta y) \Delta y
	\end{align}
	где $\alpha_1(\Delta x, \Delta y) \to 0$, $\beta_1(\Delta x, \Delta y) \to 0$ при $\Delta x \to 0$ и $\Delta y \to 0$.
	\begin{align} \label{eq:2.4.4}
		\Delta v(x_0, y_0) = \frac{\partial u(x_0, y_0)}{\partial x} \Delta x + \frac{\partial v(x_0, y_0)}{\partial y} + \alpha_2(\Delta x, \Delta y) \Delta x + \beta_2(\Delta x, \Delta y) \Delta y
	\end{align}
	где $\alpha_2(\Delta x, \Delta y) \to 0$, $\beta_2 (\Delta x, \Delta y) \to 0$ при $\Delta x \to 0$ и $\Delta y \to 0$.

	Подставим $\Delta u(x_0, y_0)$ и $\Delta v(x_0, y_0)$ из выражений \cref{eq:2.4.2,eq:2.4.4} в \cref{eq:2.4.1}, кроме двух последних слагаемых $\alpha(\Delta u, \Delta v) \Delta u$ и $\beta(\Delta u, \Delta v) \Delta v$, иначе получается громоздкие выражения:
	\begin{multline} \label{eq:2.4.5}
		\Delta z(u_0, v_0) = \frac{\partial f(u_0, v_0)}{\partial u} \bigg(\frac{\partial u(x_0, y_0)}{\partial x} \Delta x + \frac{\partial u(x_0, y_0)}{\partial y} \Delta y + \\ + \alpha_1 (\Delta x, \Delta y) \Delta x + \beta_1(\Delta x, \Delta y) \Delta y\bigg)
	\end{multline}
	Соберем коэффициент при $\Delta x$ и $\Delta y$ и учтем, что $u_0 = u(x_0, y_0)$ и $v_0 = v(x_0, y_0)$ в левой части выражения (\ref{eq:2.4.5}).
\end{tbox*}

\begin{tbox}{Следствия}
	Полученные формулы можно обобщить на любое количество промежуточных аргументов и независимых переменных. Пусть задана по $y$.
	\begin{align}
		y = f(u_1(x_1, x_2, ..., x_k), u_2(x_1, x_2, ..., x_k), ..., u_n(x_1, x_2, ..., x_k))
	\end{align}
	$n$ -- промежуточных аргументов $u_1, u_2, ..., u_n$ и $k$ независимых переменных $x_1, x_2, ..., x_k$.

	Тогда частные производные сложной функции по независимой переменным будет вычисляться по формуле:
	\begin{align}
		\boxed{\frac{\partial y}{\partial x_i} = \frac{\partial f}{\partial u_1} \times \frac{\partial u_1}{\partial x_i} + \frac{\partial f}{\partial u_2} \times \frac{\partial u_2}{\partial x_i} + ... + \frac{\partial f}{\partial u_k} \times \frac{\partial u_k}{\partial x_i} \quad (i = 1, ..., k)}
	\end{align}

	При вычислении частных производных от сложной функции \uline{нужно запомнить}, что от \uline{функции $f$} всегда берутся \uline{производные по промежуточным  переменным}, а промежуточные аргументы дифференцируются по независимым переменных $x$ и $y$.\\

	Полученные формулы являются обобщениями производной сложной функции одной переменной.
	\begin{align*}
		y = f(u(x)) \quad &\Rightarrow \quad \frac{d y}{d x} = \frac{d f}{d u} \times \frac{d u}{d x}\\
		z = f(u(x,y)) \quad &\Rightarrow \quad \frac{\partial z}{\partial x} = \frac{d  f}{d  u} \cdot \frac{\partial u}{\partial x}, \, \frac{\partial z}{\partial y} = \frac{d  f}{d  u} \cdot \frac{\partial u}{\partial y}\\
		z = f(u(x,y), v(x,y)) \quad &\Rightarrow \quad \frac{\partial z}{\partial x} = \frac{\partial f}{\partial u} \cdot \frac{\partial u}{\partial x} + \frac{\partial f}{\partial v} \cdot \frac{\partial v}{\partial x}, \\
		& \qquad \frac{\partial z}{\partial y} = \frac{\partial f}{\partial u} \cdot \frac{\partial u}{\partial y} + \frac{\partial f}{\partial v} \cdot \frac{\partial v}{\partial y} \\
		z = f(u(x,y), v(x,y), t(x,y)) \quad &\Rightarrow \quad \frac{\partial z}{\partial x} = \frac{\partial f}{\partial u} \cdot \frac{\partial u}{\partial x} + \frac{\partial f}{\partial v} \cdot \frac{\partial v}{\partial x} + \frac{\partial f}{\partial t} \cdot \frac{\partial t}{\partial x}
	\end{align*}
	\begin{align}
		\frac{\partial z(x_0, y_0)}{\partial y} = \frac{\partial f(u_0, v_0)}{\partial u} \cdot \frac{\partial u(x_0, y_0)}{\partial y} + \frac{\partial f(u_0, v_0)}{\partial v} \cdot \frac{\partial v(x_0, y_0)}{\partial y}
	\end{align}
\end{tbox}
	\subsection{Производные и дифференциалы высших порядков функции многих переменных} \label{sec:2.5}

\begin{tbox}{Частные производные 2 и 3 порядков}
	Пусть $y = f(x_1, x_2, \dots, x_k)$ имеет частные производные 1-го порядка по всем независимым переменным:
	\begin{align} \label{eq:2.5.1}
		\frac{\partial f}{\partial x_1}, \frac{\partial f}{\partial x_2}, \dots, \frac{\partial f}{\partial x_k}.
	\end{align}

	У каждой из таких производных могут существовать частные производные 1-го порядка по всем независимым переменным, которые называются \textbf{частными производными 2-го порядка}:
	\begin{align} \label{eq:2.5.2}
		\frac{\partial}{\partial x_j} \left( \frac{\partial f}{\partial x_i} \right) = \frac{\partial^2 f}{\partial x_i^2} = f^{\prime\prime}_{x_i x_i} = f^{\prime\prime}_{x_i^2}, \quad (i = 1, \dots, k).
	\end{align}

	Аналогичным образом определяются \textbf{частные производные 3-го порядка}:
	\begin{align}\label{eq:2.5.3}
		\frac{\partial}{\partial x_i} \left( \frac{\partial^2 f}{\partial x_i^2} \right) = \frac{\partial^3 f}{\partial x_i^3} = f^{\prime\prime\prime}_{x_i x_i x_i} = f^{\prime\prime\prime}_{x_i^3} \quad (i = 1,...,k)
	\end{align}
\end{tbox}

\begin{tbox}{Смешанная частная производная 2 и 3 порядков}
	Частные производные, взятые по разным независимым переменным, называются \textbf{смешанными частными производными 2-го порядка}:

	\begin{align} \label{eq:2.5.4}
		\frac{\partial}{\partial x_j} \left(\frac{\partial f}{\partial x_i}\right) = \frac{\partial^2 f}{\partial x_j x_i} = f^{\prime\prime}_{x_i x_j} (i \neq j, \, i = 1,...,k, \, j = 1,...,k)
	\end{align}

	\textbf{Смешанные частные производные 3-го порядка} вычисляются по двум независимым переменным, либо по трем и т.д. Таким образом, можно получать частные производные любого порядка.
	\begin{equation}\label{eq:2.5.5}
		\begin{cases}
			\displaystyle \frac{\partial}{\partial x_j} \left( \frac{\partial^2 f}{\partial x_i^2} \right) = \frac{\partial^3 f}{\partial x_i^2 \partial x_j}, \quad f^{(3)}_{x_i x_i x_j}, \quad &(i \neq j, \, i = 1,...,k, \, j = 1,...,k) \\
			\displaystyle \frac{\partial}{\partial x_n} \left( \frac{\partial^2 f}{\partial x_i \partial x_j} \right) = \frac{\partial^3 f}{\partial x_n \partial x_i \partial x_j} = f^{\prime\prime\prime}_{x_j x_i x_n}, &(i \neq j \neq n, \, i, j, n = 1,...,k)
		\end{cases}
	\end{equation}
\end{tbox}

\begin{tbox}{Теорема о равенстве смешанных производных}
	Пусть функция \(y = f(\vec{x})\) $n$ --- раз дифференцируема в точке \(\vec{x}_0 \in E \subset \mathbb{R}\) в этой точке значение любой смешанной частной производной не зависит от порядка, в котором проводится последовательнее дифференцирование. (без доказательства)
\end{tbox}

\begin{tbox}{Пример равенства смешанных производных}
	\textbf{Пример:} $z = f(x, y)$

	\(z_{xy} = z_{yx}\) -- смешанные частные производные 2 порядка.

	\(z_{xxy} = z_{xyx} = z_{yxx}, \quad z_{xyy} = z_{yxy} = z_{yyx}\) -- смешанные частные производные 3 порядка.
\end{tbox}

\begin{tbox}{Дифференциалы высших порядков}
	Пусть функция $y = f(\vec{x}) = f(x_1, x_2, \dots, x_k)$ дифференцируема, и её первый дифференциал равен:
	\begin{align}\label{eq:2.5.6}
		dy = \frac{\partial f(\vec{x})}{\partial x_1} dx_1 + \frac{\partial f(\vec{x})}{\partial x_2} dx_2 + \dots + \frac{\partial f(\vec{x})}{\partial x_k} dx_k.
	\end{align}

	Тогда:
	\begin{gather*}
		d(dy) = d^2 y, \quad \text{дифференциал 2-го порядка,}\\
		d(d^2 y) = d^3 y, \quad \text{дифференциал 3-го порядка,}\\
		\cdots\\
		d(d^{n-1} y) = d^n y, \quad \text{дифференциал $n$-го порядка.}
	\end{gather*}

	В случае независимых переменных можно вывести общую \textbf{формулу для дифференциалов $n$-го порядка}. Для частного случая, функции двух переменных $z = f(x, y)$:
	\begin{align} \label{eq:2.5.7}
		dz = \frac{\partial f}{\partial x} dx + \frac{\partial f}{\partial y} dy, \quad\text{где $dx = \Delta x$, $dy = \Delta y$}
	\end{align}
\end{tbox}

\begin{tbox}{Дифференциальные операторы}
	Рассмотрим дифференциалы операторы $\frac{\partial}{\partial x}, \frac{\partial}{\partial y}$, определяющие частные производные. Дифференциальные операторы \uline{всегда действуют} на функцию, перед которой они стоят.
	\begin{align} \label{eq:2.5.8}
		\frac{\partial}{\partial x} \cdot f = \frac{\partial f}{\partial x} \qquad \frac{\partial}{\partial y} \cdot f = \frac{\partial f}{\partial y}
	\end{align}

	Запишем выражения (\ref{eq:2.5.7}) с помощью введенных дифференциальных операторов.
	\begin{align} \label{3.9}
		dz = \frac{\partial f}{\partial x} dx + \frac{\partial f}{\partial y} dy
		= \left( \frac{\partial}{\partial x} dx + \frac{\partial}{\partial y} dy \right) \cdot f
	\end{align}
\end{tbox}

\begin{tbox}{Дифференциал 2-го порядка}
	Вычислим по определению дифференциал 2-го порядка:
	\begin{align} \label{eq:2.5.10*}
		d^2 z = d(dz) = d \left( \frac{\partial z}{\partial x}\right) dx + d\left(\frac{\partial z}{\partial y}\right) dy
	\end{align}

	При вычислении дифференциалов $dx$ и $dy$ независимых переменных считаем константными ($dx = \Delta x$, $dy = \Delta y$), поэтому вычисляем дифференциалы только от функций $\frac{\partial z}{\partial x}$ и $\frac{\partial z}{\partial y}$ по формуле (\ref{eq:2.5.7}):
	\begin{equation}
		\begin{cases}
			\displaystyle d \left( \frac{\partial f}{\partial x} \right) = \frac{\partial}{\partial x} \left( \frac{\partial f}{\partial x} \right) dx + \frac{\partial}{\partial y} \left( \frac{\partial f}{\partial x} \right) dy
			= \frac{\partial^2 f}{\partial x^2} dx - \frac{\partial^2 f}{\partial x \partial y} dy\\
			\displaystyle d \left( \frac{\partial f}{\partial y} \right) = \frac{\partial}{\partial x} \left( \frac{\partial f}{\partial y} \right) dx + \frac{\partial}{\partial y} \left( \frac{\partial f}{\partial y} \right) dy = \frac{\partial^2 f}{\partial y \partial x} dx - \frac{\partial^2 f}{\partial y^2} dy
		\end{cases}
	\end{equation}

	\begin{align*}
		d^2 f = \left( \frac{\partial^2 f}{\partial x^2} dx - \frac{\partial^2 f}{\partial x \partial y} dy \right) dx
		+ \left( \frac{\partial^2 f}{\partial x \partial y} dx - \frac{\partial^2 f}{\partial y^2} dy \right) dy = &\\
		= -\frac{\partial^2 f}{\partial x^2} dx^2 + \frac{\partial^2 f}{\partial y \partial x} dx \cdot dy + \frac{\partial^2 f}{\partial x \partial y} dy \cdot dx - \frac{\partial^2 f}{\partial y^2} dy^2 = &\\
		= \frac{\partial^2 f}{\partial x^2} dx^2 + 2\frac{\partial^2 f}{\partial x \partial y} dx dy + \frac{\partial^2 f}{\partial y^2} dy^2.&
	\end{align*}

	Получаем формулу дифференциала 2 порядка для формулы $z = f(x, y)$.
	\begin{align} \label{eq:2.5.11}
		d^2 z = \frac{\partial^2 f}{\partial x^2} d x^2 + 2 \frac{\partial^2 f}{\partial x \partial y} d x \, d y + \frac{\partial^2 f}{\partial y^2} d y^2
	\end{align}
\end{tbox}

\begin{tbox}{Операторная запись дифференциалов}
	Запишем эту формулу в операторном виде:
	\begin{align*}
		d^2 z = \frac{\partial^2 f}{\partial x^2} dx^2 + 2 \frac{\partial^2 f}{\partial x \partial y} dx dy + \frac{\partial^2 f}{\partial y^2} dy^2
		= \left( \frac{\partial}{\partial x} dx + \frac{\partial}{\partial y} dy \right)^2 \cdot f.
	\end{align*}
	так как $\frac{\partial}{\partial x}$ и $\frac{\partial}{\partial y}$ — дифференциальные операторы и частные производные одновременно, то:
	\begin{align*}
		\frac{\partial}{\partial x} \cdot \frac{\partial}{\partial y} \cdot f = \frac{\partial}{\partial x} \cdot \frac{\partial f}{\partial y} = \frac{\partial^2 f}{\partial x \, \partial y} = \frac{\partial^2}{\partial x \, \partial y} f,
	\end{align*}
	где $\frac{\partial^2}{\partial x \, \partial y}$ — дифференциальный оператор 2-го порядка.
	\begin{align} \label{eq:2.5.12}
		\boxed{d^2 z = \left( \frac{\partial}{\partial x} dx + \frac{\partial}{\partial y} dy \right)^2 f}
	\end{align}
	\vspace{-1em}
	\begin{align} \label{eq:2.5.13}
		\boxed{d^3 z = \left( \frac{\partial}{\partial x} dx + \frac{\partial}{\partial y} dy \right)^3 f}
	\end{align}
	\vspace{-1em}
	\begin{align} \label{eq:2.5.14}
		\boxed{d^n z = \left( \frac{\partial}{\partial x} dx + \frac{\partial}{\partial y} dy \right)^n f}, \quad n = 1, 2, 3, \dots
	\end{align}
\end{tbox}

\begin{tbox}{Обобщение на функции многих переменных}
	Полученную по формуле (\ref{eq:2.5.14}) систему обобщают на дифференциалы высших порядков функции $f$ со всеми числами независимых переменных $y = f(x_1, x_2, \dots, x_k)$:
	\begin{align*}
		d y = \frac{\partial f}{\partial x_1} d x_1 + \frac{\partial f}{\partial x_2} d x_2 + \cdots + \frac{\partial f}{\partial x_k} d x_k = \left(\frac{\partial}{\partial x_1} d x_1 + \cdots + \frac{\partial}{\partial x_k} d x_k \right) \cdot f
	\end{align*}
	\vspace{-2em}
	\begin{align*}
		d^n y = \left(\frac{\partial}{\partial x_1} d x_1 + \frac{\partial}{\partial x_2} d x_2 + \cdots + \frac{\partial}{\partial x_k} d x_k \right)^n \cdot f
	\end{align*}
\end{tbox}

	
\subsection{Производные второго порядка сложной функции}

\begin{tbox}{Производные второго порядка сложной функции}
	Выведем формулы производных второго порядка сложной функции двух промежуточных аргументов $u$ и $v$ и двух независимых переменных $x$ и $y$: \(z = f(u(x,y), v(x,y))\).
	\begin{align} \label{eq:2.5.15}
		z^{\prime}_x = f_u^{\prime}(u, v) u^{\prime}_x(x,y) + f^{\prime}_v(u, v)v^{\prime}_x(x,y)
	\end{align}

	В формуле (\ref{eq:2.5.15}), выведенной в \cref{sec:1.5} частные производные по промежуточным аргументам $f_u^{\prime}(u, v)$ и $f_v^{\prime}(u, v)$ являются функциями, зависящими от $u$ и $v$. Поэтому в (\ref{eq:2.5.15}) введем обозначения:
	\begin{align} \label{eq:2.5.16}
		g(u, v) = f^{\prime}_u(u,v) \quad \text{и} \quad h(u, v) = f^{\prime}_v(u,v)
	\end{align}
	и перепишем выражение (\ref{eq:2.5.15}) в виде:
	\begin{align} \label{eq:2.5.17}
		z_x^{\prime} = g(u,v) \cdot u^{\prime}_x(x,y) + h(u,v) \cdot v_x^{\prime}(x,y) \quad |\textstyle\frac{\partial}{\partial x}
	\end{align}
	и вычислим частную производную по $x$.
	\begin{gather*}
		z^{\prime\prime}_x = (g(u,v) \cdot u_x^{\prime}(x,y) + h(u,v) \cdot v_x^{\prime}(x,y))_x^{\prime} =\\
		(g(u, v))^{\prime}_x \cdot u_x^{\prime}(x, y) + g(u,v) \cdot (u_x^{\prime}(x,y))^{\prime}_x + \\ + (h(u,v))^{\prime}_x \cdot v_{x}^{\prime}(x, y) + h(u, v) \cdot (v_x^{\prime}(x, y))^{\prime}_x = \\ =
		(g(u,v))^{\prime}_x \cdot u_x^{\prime}(x,y) + g(u, v) \cdot u_{xx}^{\prime\prime}(x,y) + (h(u,v))_x^{\prime} \cdot v_x^{\prime}(x,y) + h(u, v) \cdot v_{xx}^{\prime\prime}(x,y)
	\end{gather*}

	Вычислим производные сложных функций $g(u, v)$ и $h(u, v)$ по $x$, используя формулу (\ref{eq:2.5.15}):
	\begin{align*}
		\text{Подставим в формулу $z_{xx}^{\prime\prime} \quad$}
		\begin{cases}
			(g(u,v))_x^{\prime} = g_u^{\prime}(u, v) \cdot u_x^{\prime}(x, y) + g_v^{\prime}(u, v) \cdot v_x^{\prime}(x,y)\\
			(h(u,v))_x^{\prime} = h_u^{\prime}(u, v) \cdot u_x^{\prime}(x, y) + h_v^{\prime}(u, v) \cdot v_x^{\prime}(x,y)
		\end{cases}
	\end{align*}
	\begin{gather*}
		z_{xx}'' = \big(g_u'(u,v)u_x'(x,y) + g_x'(u, v) v_x'(x,y)\big) \cdot u_x'(x,y) + g(u,v) u_{xx}''(x,y) + \\
		+ \big(h_u'(u,v) u_x'(x,y) + h_x'(u,v) v_x'(x,y)\big) \cdot v_x'(x,y) + h(u,v) \cdot v_{xx}''(x,y) = \\
		\text{подставляем выражения для $g(u,v)$ и $h(u,v)$ из (\ref{eq:2.5.16})}\\
		= \big((f_u')_u' u_x' + (f_u')_v' v_x'\big) \cdot u_x' + f_u' u_{xx}'' + \big((f_v')u_x' + (f_v')_u' u_x'\big) \cdot v_x' + f_v' v_{xx}'' = \\
		= f_{uu}'' u_x'^2 + 2 f_{uv}'' u_x' v_x' + f_{vv}'' v_x'^2 + f_u' u_{xx}'' + f_v' v_{xx}''.
	\end{gather*}
	\begin{align} \label{eq:2.5.19}
		\boxed{z_{xx}'' = f_{uu}'' u_x'^2 + 2 f_{uv}'' u_x' v_x' + f_{vv}'' v_x'^2 + f_u' u_{xx}'' + f_v' v_{xx}''}
	\end{align}
\end{tbox}

\begin{tbox}{Производная второго порядка по $y$}
	Так как частная производная сложной функции первого порядка по $y$ имеет аналогичный вид выражению (\ref{eq:2.5.15}). \(z_y' = f_u' u_y' + f_v' v_y'\), то частную производную второго порядка по $y$, запишем аналогично (\ref{eq:2.5.19}), заменив $x$ на $y$.
	\begin{align} \label{eq:2.5.20}
		\boxed{z_{yy}'' = f_{uu}'' u_y'^2 + 2 f_{uv}'' u_y' v_y' + f_{vv}'' v_y'^2 + f_u' u_{yy}'' + f_v' v_{yy}''}
	\end{align}
\end{tbox}

\begin{tbox}{Смешанная производная $z''_{xy}$}
	Выведем смешанную производную $z''_{xy}$. Для этого выражение (\ref{eq:2.5.15}) для $z_x^{\prime}$ перепишем с учетом (\ref{eq:2.5.16}).
	\begin{gather*}
		z'_x = g(u,v) u'_x + h(u,v) v'_x, \quad \frac{\partial}{\partial y}\\
		z''_{xy} = \big(g(u,v) u'_x(x,y) + h(u,v) v'_x(x,y)\big)'_y =\\
		= \big(g(u,v)\big)'_y u'_x(x,y) + \left(u_x^{\prime}(x,y)\right)^{\prime}_y g(u, v) + \big(h(u,v)\big)'_y v'_x(x,y) + h(u, v) \left(v_x^{\prime}(x,y)\right)^{\prime}_y.
	\end{gather*}

	Распишем производные по $y$ от сложенных функций $g(u,v)$ и $h(u,v)$:
	\begin{gather*}
		(g(u,v))'_y = g'_u(u,v) u'_y(x,y) + g'_v(u,v) v'_y(x,y),\\
		(h(u,v))'_y = h'_u(u,v) u'_y(x,y) + h'_v(u,v) v'_y(x,y).
	\end{gather*}

	Подставляем в $z''_{xy}$:
	\begin{gather*}
		z''_xy = (g_u'(u,v) u_y'(x,y)+g_v'(u,v)v_y'(x,y)) \cdot u_x'(x, y) + g(u, v) u_{xy}''(x,y) + \\ +(h'_u(u, v) u_y'(x,y) + h_v'(u, v) v_y'(x,y)) \cdot v_x'(x,y) + h(u, v) v_{xy}''(x,y) = \\
		=(g_u' u_y' + g_v' v_y') u_x' + g(u, v) u_{xy}'' + (h_u' u_y' + h_v' v_y') v_x' + h(u, v) \cdot v_{xy}'' = ...
	\end{gather*}

	Подставим обозначения (\ref{eq:2.5.16}):
	\begin{gather*}
		... = \left((f_u')_u' u_y' + (f_u')_v' v_y'\right) u_x' + f_u' u_{xy}'' + \left((f_v')_u' u_y' + (f_v')_v' v_y'\right) v_x' + f_v' u_{xy}'' = \\
		= (f_{uu}'' u_y' + f_{uv}'' v_y') u_x' + f_u'' u_{xy}'' + (f_{uv}'' u_y' + f_{vv}'' v_y') v_x' + f_v' v_{xy}'' = \\
		= f_{uu}'' u_y' u_x' + f_{uv}'' v_y' u_x' + f_u'' u_{xy}'' + f_{uv}'' u_y' v_x' + f_{vv}'' v_y' v_x' + f_v' v_{xy}'' = \\
		= f_{uu}'' u_x' u_y' + f_{uv}''(u_x' v_y' + u_y' v_x') + f_{vv}'' v_x' v_y' + f_u' u_{xy}'' + f_v' v_{xy}''
	\end{gather*}
	\begin{align} \label{eq:2.5.18}
		\boxed{z''_xy = f_{uu}'' u_x' u_y' + f_{uv}''(u_x' v_y' + u_y' v_x') + f_{vv}'' v_x' v_y' + f_u' u_{xy}'' + f_v' v_{xy}''}
	\end{align}
\end{tbox}
	\subsection{Дифференциалы сложной функции} \label{part:2.6}

\begin{tbox}{Дифференциал сложной функции}
	Рассмотрим сложную функцию двух промежуточных аргументов $u$ и $v$ и двух независимых переменных $x$ и $y$ и вычислим дифференциал этой функции.
	\begin{align*}		\label{eq:4.1}
		z = f(u(x, y), v(x, y))
	\end{align*}
	Вспомним формулу дифференциала функции $z = z(x,y)$ независимых переменных $x$ и $y$.
	\begin{align}
		dz = \frac{\partial z}{\partial x} dx + \frac{\partial z}{\partial y} dy
	\end{align}
	Вычислим частные производные \(\frac{\partial z}{\partial x}\) и \(\frac{\partial z}{\partial y}\) для сложной функции:
	\begin{align*}
		\frac{\partial z}{\partial x} = \frac{\partial f}{\partial u} \times \frac{\partial u}{\partial x} + \frac{\partial f}{\partial v} \times \frac{\partial v}{\partial x}; \quad \frac{\partial z}{\partial y} = \frac{\partial f}{\partial u} \times \frac{\partial u}{\partial y} + \frac{\partial f}{\partial v} \times \frac{\partial v}{\partial y}
	\end{align*}
	и подставим в \cref{eq:4.1}.
	\begin{align*}
		dz = \left(\frac{\partial f}{\partial u} \times \frac{\partial u}{\partial x} + \frac{\partial f}{\partial v} \times \frac{\partial v}{\partial x}\right) dx + \left(\frac{\partial f}{\partial u} \times \frac{\partial u}{\partial y} + \frac{\partial f}{\partial v} \times \frac{\partial v}{\partial y}\right)dy
	\end{align*}
	Раскроем скобки и сгруппируем слагаемые при $\frac{\partial f}{\partial u}$ и $\frac{\partial f}{\partial v}$:
	\begin{equation}	\label{eq:4.2}
		\begin{split}
			dz = \left(\frac{\partial f}{\partial u}  \frac{\partial u}{\partial x} dx + \frac{\partial f}{\partial u}  \frac{\partial u}{\partial y} dy \right) + \left(\frac{\partial f}{\partial v}  \frac{\partial v}{\partial x}dx + \frac{\partial f}{\partial v}  \frac{\partial v}{\partial y} dy\right) = \\
			\frac{\partial f}{\partial u} \underbrace{\left(\frac{\partial u}{\partial x} dx + \frac{\partial u}{\partial y} dy\right)}_{du} + \frac{\partial f}{\partial v}  \underbrace{\left(\frac{\partial v}{\partial x} dx + \frac{\partial v}{\partial y} dy\right)}_{dv}
		\end{split}
	\end{equation}
	Так как $u = u(x,y)$ и $v = v(x,y)$ - функции переменных $x$ и $y$, тогда из дифференциалы равны:
	\begin{equation*}
		du = \frac{\partial u}{\partial x} dx + \frac{\partial u}{\partial y} dy \quad \text{и} \quad dv = \frac{\partial v}{\partial x} dx + \frac{\partial v}{\partial y} dy
	\end{equation*}
	Из выражения \cref{eq:4.2} получаем окончательный вид дифференциал $dz$:
	\begin{equation}	\label{eq:4.3}
		\boxed{dz = \frac{\partial f}{\partial u} du + \frac{\partial f}{\partial v} dv}
	\end{equation}
\end{tbox}

\begin{tbox}{Свойство инвариантности дифференциала}
	Формула первого дифференциала имеют один и тот же вид, как для функции двух независимых переменных $z = z(x,y) = f(x,y)$, так и для сложной функции двух промежуточных аргументов $z = f(u,v)$. Это свойство первого дифференциала называется \textbf{свойством инвариантности или сохранения формы дифференциала}. \\

	Отличие состоит только в том, что в выражении \cref{eq:4.1} $dx$ и $dy$ - это дифференциалы независимых переменных, т.е. $dx = \Delta x$ и $dy = \Delta y$ \textit{- const}. В формуле \cref{eq:4.3} $du$ и $dv$ - это дифференциалы независимых переменных, т.е. $dx = \Delta x$ - дифференциалы функций $u = u(x,y)$ и $v = v(x,y)$:
	\[du = u_x' dx + u_y'dy \quad \text{и} \quad dv = v_x' dx + v_y'dy\]
\end{tbox}

\begin{tbox}{Дифференциал второго порядка}
	Выведем формулу дифференциала второго порядка для сложной функции $z = f(u,v)$. По определению:
	\begin{equation*}
		d^2 z = d(dz) = d(\frac{\partial f}{\partial u} du + \frac{\partial f}{\partial v} dv)
	\end{equation*}
	Вычислим тот дифференциал, применяя правило суммы и произведения дифференциалов:
	\begin{multline} \label{eq:4.4}
		d^2 z = d\left(\frac{\partial f}{\partial u} du + \frac{\partial f}{\partial v} dv\right) = d\left(\frac{\partial f}{\partial u} du\right) + d\left(\frac{\partial f}{\partial v} dv\right) =\\=  d\left(\frac{\partial f}{\partial u}\right) du + \frac{\partial f}{\partial u} d\left(du\right) + d\left(\frac{\partial f}{\partial v}\right) dv + \frac{\partial f}{\partial v} d(dv)
	\end{multline}

	По определению $d(du) = d^2 u$ и $d(dv) = d^2 v$ - это дифференциалы 2-ого порядка от функций $u = u(x,y)$ и $v = v(x,y)$.

	Дифференциалы $d\left(\frac{\partial f}{\partial u}\right)$ и $d\left(\frac{\partial f}{\partial v}\right)$ вычислим, применяя формулу \cref{eq:4.3}.
	\begin{multline*}
		d^2 z = \left(\frac{\partial^2 f}{\partial u^2} du + \frac{\partial^2 f}{\partial v \partial u} dv\right) du + \frac{\partial f}{\partial u} d^2 u + \left(\frac{\partial^2 f}{\partial u \partial v} du + \frac{\partial ^2 f}{\partial v^2}dv\right)dv + \frac{\partial f}{\partial v} d^2 v = \\
		= \frac{\partial^2 f}{\partial u^2} du^2 + \frac{\partial^2 f}{\partial v \partial u} dv du + \frac{\partial^2 f}{\partial u \partial v} du dv + \frac{\partial^2 f}{\partial v^2} dv^2 + \frac{\partial f}{\partial u} d^2 u + \frac{\partial f}{\partial v} d^2 v = \\
		= \frac{\partial^2 f}{\partial u^2} du^2 + 2 \frac{\partial^2 f}{\partial u \partial v} du dv + \frac{\partial^2 f}{\partial v^2} dv^2 + \frac{\partial f}{\partial u} d^2 u + \frac{\partial f}{\partial v} d^2 v
	\end{multline*}

	Дифференциал 2-ого порядка для сложной функции имеет вид:
	\begin{equation} \label{eq:4.5}
		\boxed{d^2 z = \frac{\partial^2 f}{\partial u^2} du^2 + 2 \frac{\partial^2 f}{\partial u \partial v} du dv + \frac{\partial^2 f}{\partial v^2} dv^2 + \frac{\partial f}{\partial u} d^2 u + \frac{\partial f}{\partial v} d^2 v}
	\end{equation}
\end{tbox}

\begin{tbox}{Сравнение дифференциалов}
	Формула $d^2 z$ для функции $z = f(x,y)$:
	\begin{equation} \label{eq:4.6}
		\boxed{d^2 z = \frac{\partial^2 f}{\partial x^2} dx^2 + 2 \frac{\partial^2 f}{\partial x \partial y} dx dy + \frac{\partial^2 f}{\partial y^2} dy^2}
	\end{equation}

	Сравнение формул \cref{eq:4.5,eq:4.6} показывает, что форма дифференциала второго порядка сложной функции отличается от случая независимых переменных. В этом случае говорят о нарушении инвариантности формы высших дифференциалов сложной функции и в этом случае нельзя вывести общую формулу вычисления дифференциалов высших порядков.
\end{tbox}

\begin{tbox}{Формулы}
	\(z = f(x,y) \quad x, y - \text{независимые переменные}\);\\
	\[dz = \frac{\partial u}{\partial x} dx + \frac{\partial u}{\partial y} dy\]
	\[d^2 z = \frac{\partial f}{\partial x} dx^2 + 2 \frac{\partial^2 f}{\partial x \partial y} dxdy + \frac{\partial^2 f}{\partial y^2} dy\]
	$z = f(u,v)$, $u = u(x,y)$, $v = v(x,y)$ - зависимые аргументы:
	\[dz = \frac{\partial f}{\partial u} du + \frac{\partial f}{\partial v} dv\]
	\[d^2 z = \frac{\partial^2 f}{\partial u^2} du^2 + 2 \frac{\partial^2 f}{\partial u \partial v} dudv + \frac{\partial^2 f}{\partial v^2} dv^2 + \frac{\partial f}{\partial u} d^2 u + \frac{\partial f}{\partial v} d^2 v\]

	\begin{center}
		\textbf{Частный случай:}
	\end{center}
	$z = f(u)$, $u = u(x,y)$ -- промежуточный аргумент:
	\[dz = \frac{df}{du}du = f_u' du\]
	\[d^2 z = \frac{d^2 f}{du^2} du^2 + \frac{df}{du} d^2 u = f_{uu}'' du^2 + f_u' d^2 u\]
\end{tbox}
	\subsection{Дифференцирование неявно заданной функции} \label{part:2.8}

Рассмотрим неявно заданную функцию $z = z(x,y)$ двух независимых переменных.\\

\textbf{Определение:} Уравнение \cref{eq:4.7} определяет функцию $z = z(x,y)$ как неявно заданную функцию двух переменных, если при подстановке этой функции в уравнение \cref{eq:4.7} оно становится тождеством:
\begin{equation} \label{eq:4.7}
	F(x,y,z) = 0
\end{equation}
\begin{equation} \label{eq:4.8}
	F(x,y,z(x,y)) \equiv 0
\end{equation}

Неявно заданную функцию из уравнения \cref{eq:4.7} можно найти при выполнении условий следующей теоремы.

\begin{tbox}{Теорема о существовании и дифференцируемости неявно заданной функции (без доказательства)}
	Пусть функция $F(x,y,z)$ и все ее частные производные $\frac{\partial F}{\partial x}$, $\frac{\partial F}{\partial y}$ и $\frac{\partial F}{\partial z}$ определены и непрерывны в некоторой окрестности точки $M_0(x_0,y_0, z_0)$, причем $F(M_0) = F(x_0, y_0, z_0) = 0$, а частная производной $\frac{\partial F(M_0)}{\partial z} \neq 0$.\\

	Тогда $\exists$ окрестность точки $M_0$ такая, что уравнение \cref{eq:4.7} определяет неявно заданную функцию $z = z(x,y)$, непрерывную и дифференцируемую в точке $(x_0, y_0)$, причем $z_0 = z(x_0, y_0)$.
\end{tbox}

Для нахождения частных производных подставляем в уравнение \cref{eq:4.7} функцию $z = z(x,y)$ и получаем тождество \cref{eq:4.8}, которое дифференцируем по независимым переменных $x$ и $y$.
\begin{gather*}
	F(x,y,z(x,y)) \equiv 0 \quad | \frac{\partial }{\partial x}\\
	\frac{\partial F}{\partial x} + \frac{\partial F}{\partial z} \cdot \frac{\partial z}{\partial x} = 0, \text{ откуда находим $\frac{\partial z}{\partial x}$}
\end{gather*}
\begin{equation} \label{eq:4.9}
	-\frac{\partial F}{\partial x} = \frac{\partial F}{\partial z} \cdot \frac{\partial z}{\partial x} \quad \Rightarrow \quad \boxed{\frac{\partial z}{\partial x} = -\frac{\frac{\partial F}{\partial x}}{\frac{\partial F}{\partial z}}}
\end{equation}
Аналогично, дифференцируя тождество \cref{eq:4.8} по $y$, найдем $\frac{\partial z}{\partial y}$:
\begin{equation} \label{eq:4.10}
	\frac{\partial F}{\partial y} + \frac{\partial F}{\partial z} \cdot \frac{\partial z}{\partial y} = 0 \quad \Rightarrow \quad \boxed{\frac{\partial z}{\partial y} = - \frac{\frac{\partial F}{\partial y}}{\frac{\partial F}{\partial z}}}
\end{equation}
Для нахождения частных производных неявно заданной функции формулы \cref{eq:4.9,eq:4.10} применять не будем а для каждого примера будем проводить данную процедуру.
\subsubsection*{Пример}
Найти $\frac{\partial z}{\partial x}$, $\frac{\partial z}{\partial y}$, если $z^3 - xz + y = 0$.
\begin{gather*}
	F(x,y,z) = z^3 - xz + y\\
	z^3(x,y) - xz(x,y) + y = 0 \quad \Big|\frac{\partial}{\partial x} \quad \Big|\frac{\partial}{\partial y}\\
	3z^2(x,y) z_x' - z(x,y) - x z_x' = 0 \quad (3z^2 - x) z_x' = 0\\
	z_x' = \frac{z}{3z^2 - x} \quad \text{($3z^2 - x \neq 0$, т.к. по условию теоремы $\frac{\partial F}{\partial z} = 3x^2 - x \neq 0$)}\\
	\begin{cases}
		3z^2(x,y) z_y' - xz_y' + 1 = 0\\
		(3z^2 - x)z_y' = -1
	\end{cases}
	\Rightarrow\quad z_y' = -\frac{1}{3z^2 - x}
\end{gather*}
Для нахождения частной производной 2-ого порядка дифференцируем найденные производные:
\begin{align*}
	z_x' &= \frac{z(x,y)}{3z^2(x,y) - x} \quad \left| \frac{\partial}{\partial x} \right. \\
	z_{xx}''
	&= \frac{z_x'(3z^2 - x) - z(6z \cdot z_x' - 1)}{(3z^2 - x)^2}
	= \frac{z - \dfrac{6z^3}{3z^2 - x} + z}{(3z^2 - x)^2}
	\quad \text{(подстановка $z_x'$)} \\
	&= \frac{2z - \dfrac{6z^3}{3z^2 - x}}{(3z^2 - x)^2}
	= \frac{2z(3z^2 - x) - 6z^3}{(3z^2 - x)^3} \\
	&= \frac{6z^3 - 2xz - 6z^3}{(3z^2 - x)^3}
	= -\frac{2xz}{(3z^2 - x)^3}\\
	z_y' &= -\frac{1}{3z^2(x,y) - x} \quad \left| \frac{\partial}{\partial y} \right. \quad \left| \frac{\partial}{\partial x} \right. \\
	z_{yy}''
	&= \frac{6z \cdot z_y'}{(3z^2 - x)^2}
	= \frac{6z \cdot \left(-\dfrac{1}{3z^2 - x}\right)}{(3z^2 - x)^2}
	= -\frac{6z}{(3z^2 - x)^3} \\
	z_{yx}''
	&= \frac{6z \cdot z_x' - 1}{(3z^2 - x)^2}
	= \frac{\dfrac{6z^2}{3z^2 - x} - 1}{(3z^2 - x)^2}
	= \frac{6z^2 - (3z^2 - x)}{(3z^2 - x)^3} \\
	&= \frac{3z^2 + x}{(3z^2 - x)^3}
\end{align*}


	\input{source/2_9}
	\subsection{Замена переменных в дифференциальных уравнениях или выражениях, содержащих частные производные}

\begin{tbox}{Замена переменных в дифференциальных уравнениях}
	Пусть задана некоторое дифференциальное уравнение или выражение, содержащие независимые переменные $x$ и $y$, функция $z = z(x,y)$ и частные производные этой функции:
	\begin{align} \label{eq:5.4}
		F(x,y,z, \frac{\partial z}{\partial x}, \frac{\partial z}{\partial y}, \dots) = 0
	\end{align}
	В данном уравнении нужно ввести новые независимые переменные $u$ и $v$ и новую функцию $W = W(u,v)$, т.е. сделать замену переменных.

	Пусть замена переменных осуществляется с помощью системы, когда новые переменные $u$, $v$ и $w$ выражаются через прежние (старые переменные).
	\begin{equation} \label{eq:5.5}
		\begin{cases}
			u = f(x,y,z)\\
			v = g(x,y,z)\\
			w = p(x,y,z)
		\end{cases}
	\end{equation}

	При замене переменных (\ref{eq:5.5}) уравнение (\ref{eq:5.4}) будет преобразовано к виду:
	\begin{align} \label{eq:5.6}
		F^*(u,v,w, \frac{\partial w}{\partial x}, \frac{\partial w}{\partial y}, \dots) = 0
	\end{align}

	для этого переменные $x$, $y$, $z$ выражаются из системы (\ref{eq:5.5}), а для выражения частных производных $\frac{\partial z}{\partial x}$ и $\frac{\partial z}{\partial y}$ через новые частные производные $\frac{\partial w}{\partial u}$, $\frac{\partial w}{\partial v}$ поступают следующим образом. Для этого систему (\ref{eq:5.5}) переписываем в виде, удобном для дифференцирования. Так как $z = z(x,y)$, то из системы (\ref{eq:5.5}) получаем, что $u = u(x,y)$, $v = v(x,y)$, а новая функция $w = w(u,v) = v(u(x,y), v(x,y))$. Тогда система (\ref{eq:5.5}) может быть переписана в виде:
	\begin{equation} \label{eq:5.7}
		\begin{split}
			\text{Новые переменные} \quad & \text{Старые переменные}\\
			u(x,y) =& \, f(x,y,z(x,y))\\
			v(x,y) =& \, g(x,y,z(x,y))\\
			w(u(x,y), v(x,y)) =& \, P(x,y,z(x,y))
		\end{split}
	\end{equation}
\end{tbox}

\begin{tbox}{Дифференцирование системы}
	Дифференцируем систему (\ref{eq:5.7}) по $x$:
	\begin{equation*}
		\begin{cases}
			\frac{\partial u}{\partial x} = \frac{\partial f}{\partial x} + \frac{\partial f}{\partial z} \cdot \frac{\partial z}{\partial x} \\
			\frac{\partial v}{\partial x} = \frac{\partial g}{\partial x} + \frac{\partial g}{\partial z} \cdot \frac{\partial z}{\partial x} \\
			\frac{\partial w}{\partial u} \cdot \frac{\partial u}{\partial x} + \frac{\partial w}{\partial v} \cdot \frac{\partial v}{\partial x} = \frac{\partial p}{\partial z} \cdot \frac{\partial z}{\partial x}
		\end{cases}
	\end{equation*}
	Подставляем в последнее уравнение $\frac{\partial u}{\partial x}$ и $\frac{\partial v}{\partial x}$ из первых двух уравнений:
	\begin{equation*}
		\frac{\partial w}{\partial u} \left(\frac{\partial f}{\partial x} + \frac{\partial f}{\partial z} \frac{\partial z}{\partial x}\right) + \frac{\partial w }{\partial v}\left(\frac{\partial g}{\partial x} + \frac{\partial g}{\partial z} \frac{\partial z}{\partial x}\right) = \frac{\partial p}{\partial x} + \frac{\partial p}{\partial z} \frac{\partial z}{\partial x}
	\end{equation*}
	Выражаем из получившегося уравнения $\frac{\partial z}{\partial x}$.
	\begin{equation*}
		\frac{\partial w}{\partial u} \frac{\partial f}{\partial x} + \frac{\partial w}{\partial v} \frac{\partial g}{\partial x} - \frac{\partial p}{\partial x} = \left(\frac{\partial p}{\partial z} - \frac{\partial w}{\partial u} \frac{\partial f}{\partial z} - \frac{\partial w}{\partial v} \frac{\partial g}{\partial z}\right) \frac{\partial z}{\partial x}
	\end{equation*}
	\begin{equation} \label{eq:5.8}
		\frac{\partial z}{\partial x} = \frac{\displaystyle\frac{\partial w}{\partial u} \frac{\partial f}{\partial x} + \frac{\partial w}{\partial v} \frac{\partial g}{\partial x} - \frac{\partial p}{\partial x}}{\displaystyle\frac{\partial p}{\partial z} - \frac{\partial w}{\partial u} \frac{\partial f}{\partial z} - \frac{\partial w}{\partial v} \frac{\partial g}{\partial z}}
	\end{equation}

	Аналогичным образом, дифференцируя систему \cref{eq:5.7} по $y$, найдем $\frac{\partial z}{\partial y}$.
	\begin{equation} \label{eq:5.9}
		\frac{\partial z}{\partial y} = \frac{\displaystyle\frac{\partial w}{\partial u} \frac{\partial f}{\partial y} + \frac{\partial w}{\partial v} \frac{\partial g}{\partial y} - \frac{\partial p}{\partial y}}{\displaystyle\frac{\partial p}{\partial z} - \frac{\partial w}{\partial u} \frac{\partial f}{\partial z} - \frac{\partial w}{\partial v} \frac{\partial g}{\partial z}}
	\end{equation}

	Формулы \cref{eq:5.8,eq:5.9} позволяют преобразовать частные производные $\frac{\partial z}{\partial x}$, $\frac{\partial z}{\partial y}$ в старые переменные через частные производные новых переменных $\frac{\partial w}{\partial u}$, $\frac{\partial w}{\partial v}$.
\end{tbox}

\subsubsection*{Пример замены переменных}
	Перейти к новым переменным $u$, $v$ и $w(u,v)$ в уравнении:
	\begin{equation*}
		x^2 \frac{\partial z}{\partial x} + y^2 \frac{\partial z}{\partial y} = z^2, \quad \text{если $u=x, v = \frac{1}{y} - \frac{1}{x}$,}
	\end{equation*}
	Запишем для данной замены $w = \frac{1}{z} - \frac{1}{x}$ систему \cref{eq:5.7}:
	\begin{equation*}
		\begin{cases}
			u(x) = x\\
			v(x,y) = \frac{1}{y} - \frac{1}{x}\\
			w(u(x), v(x,y)) = \frac{1}{z(x,y)} - \frac{1}{x}
		\end{cases}
		\Bigg|\frac{\partial}{\partial x} \quad \Bigg| \frac{\partial}{\partial y}
	\end{equation*}
	Дифференцируем систему по $x$:
	\begin{multline*}
		\begin{cases}
			u_x' = 1\\
			v_x' = \frac{1}{x^2}\\
			w_u' \cdot u_x' + w_v' \cdot v_x' = -\frac{1}{z^2} \cdot z_x' + \frac{1}{x^2}
		\end{cases} \Rightarrow\\\Rightarrow w_u' + \frac{1}{x^2} w_v' - \frac{1}{x^2} = - \frac{1}{z^2} z_x' \Rightarrow z_x' = z^2\left(\frac{1}{x^2} - w_u' - \frac{1}{x^2} w_v'\right)
	\end{multline*}
	Дифференцируем систему по $y$:
	\begin{equation*}
		\begin{cases}
			u_y' = 0\\
			v_y' = -\frac{1}{y^2}\\
			w_v' \cdot v_y' = -\frac{1}{z^2} \cdot z_y'
		\end{cases}
		\Rightarrow -\frac{1}{y^2} w_v' = -\frac{1}{z^2} z_y' \Rightarrow z_y' = \frac{z^2}{y^2} w_v'
	\end{equation*}
	Подставляем $z_x'$ и $z_y'$ в исходное уравнение:
	\begin{equation*}
		\begin{split}
			x^2 \cdot z^2 \left(\frac{1}{x^2} - w_u' - \frac{1}{x^2} w_v'\right) + y^2 \cdot \frac{z^2}{y^2} w_v' = z^2\\
			z^2\left(1 - \frac{1}{x^2} w_u' - w_v'\right) + z^2 w_v' = z^2 \quad \Big| : z^2\\
			1 - \frac{1}{x^2}w_u' - w_v' + w_v' = 1 \quad \Rightarrow \quad -\frac{1}{x^2} w_u' = 0 \quad \Big| \cdot (-x^2)\\
			w_u' = 0 \quad \text{или} \quad \frac{\partial w}{\partial u} = 0.
		\end{split}
	\end{equation*}
	Такой вид имеет уравнение в новых переменных.


\begin{tbox}{Замена независимых переменных}
	Если требуется заменить независимые переменные $x$ и $y$ на новые независимые переменные $u$ и $v$, а функция $z$ остается прежней, только в новых переменных она будет зависеть от $u$ и $v$, в этом случае систему (\ref{eq:5.7}) нужно записать в виде:
	\begin{equation} \label{eq:5.10}
		\begin{cases}
			u(x,y) = \, f(x,y,z(x,y))\\
			v(x,y) = \, g(x,y,z(x,y))\\
			z(u(x,y), v(x,y)) = \, z(x,y)
		\end{cases}
	\end{equation}
\end{tbox}
\newpage
\subsubsection*{Пример}
Преобразовать уравнение $x \frac{\partial z}{\partial x} + y \frac{\partial z}{\partial y} = \frac{x}{z}$, если $u = 2x - z^2$, $v = \frac{y}{2}$ -- новые независимые переменные. Запишем систему:
\begin{equation*}
	\begin{cases}
		u(x,y) = 2x - z^2(x,y)\\
		v(x,y) = \frac{y}{z(x,y)}\\
		z(u(x,y), v(x,y)) = z(x,y)
	\end{cases}
	\Bigg| \frac{\partial}{\partial x} \Rightarrow\\
	\begin{cases}
		u_x' = 2 - 2z \cdot z_x'\\
		v_x' = -\frac{y}{z^2} \cdot z_x'\\
		z_u' \cdot u_x' + z_v' \cdot v_x' = z_x'
	\end{cases} \Rightarrow
\end{equation*}
\begin{align*}
	2z_u'(1 - z \cdot z_x') - \frac{y}{z^2} z_v' z_x' = z_x'
\end{align*}

Выразим $z_x'$: \(\quad 2z_u' = \left(2z \cdot z_u' + \frac{y}{z^2} z_v' + 1\right) \cdot z_x'\)
\begin{gather*}
	z_x' = \frac{2 z_u'}{2z \cdot z_u' + \frac{y}{z^2} z_v' + 1} \qquad u_y' = -2z \cdot z_y' \qquad v_y' = \frac{z - y \cdot z_y'}{z^2} = \frac{1}{z} - \frac{y}{z^2}z_y'\\
	z_u' u_y' + z_v' v_y' = z_y' \quad \Rightarrow \quad -2z \cdot z_u' \cdot z_y' + z_v' \left(\frac{1}{z} - \frac{y}{z^2}\right) z_y' = z_y'.\\
	z_y' = \frac{\frac{1}{z} z_v'}{2z \cdot z_u' + \frac{y}{z^2} z_v' + 1} \quad \text{-- подставляем в уравнение}\\
	\frac{x \cdot 2z_y'}{2z \cdot z_u' + \frac{y}{z^2} z_v' + 1} + \frac{y \frac{1}{z} z_v'}{2z \cdot z_u' + \frac{y}{z^2}z_v' + 1} = \frac{x}{z} \quad \Big| \cdot (2z \cdot z_u' + \frac{y}{z^2} z_v' + 1)\\
	2x  z_u' + \frac{y}{z} z_v' = \frac{x}{2} 2 z  z_u' + \frac{x}{z} \frac{y}{z^2} z_v' + \frac{x}{z}\\
	2x  z_u' + \frac{y}{z} z_v' = 2x  z_u' + \frac{y}{z} \frac{x}{z^2} z_v' + \frac{x}{z}\\
	\frac{y}{z}\left(1 - \frac{x}{z^2}\right) z_v' = \frac{x}{z}\\
	u = 2x - z^2  \Rightarrow  x = \frac{u+z^2}{2} \qquad v = \frac{y}{z}\\
	v(z^2 - x^2) z_v' = xz \Rightarrow z_v' = \frac{z}{v} \frac{z^2 + u}{z^2 - u} \quad \Rightarrow \quad \boxed{\text{\textbf{Ответ: }} \frac{\partial z}{\partial v} = \frac{z}{v} \frac{z^2 + u}{z^2 - u}}
\end{gather*}


	\subsection{Производная по направлению}

\begin{tbox}{Базис в пространстве \(\mathbb{R}^k\)}
	Вектора \(\vec{e}_1 = (1,0,\dots, 0)\), \(\vec{e}_2 = (0,1,\dots, 0)\), \(\dots\), \(\vec{e}_k = (0,0,\dots, 1)\) называются \uline{базисом в пространстве \(\mathbb{R}^k\)}. Если \(\vec{x} = (x_1, x_2, \dots, x_k) \in \mathbb{R}^k\), то этот вектор можно разложить по векторам базиса следующим образом:

	\begin{equation*}
		\boxed{
			\vec{x} = x_1 \vec{e}_1 + x_2 \vec{e}_2 + \dots + x_k \vec{e}_k = \sum_{i=1}^k x_i \vec{e}_i
		}
	\end{equation*}

	Для \(\mathbb{R}^3\) часто используются обозначения:
	\[
	\vec{i} = (1,0,0), \quad \vec{j} = (0,1,0), \quad \vec{k} = (0,0,1).
	\]

	Тогда любой вектор \(\vec{a} = (x, y, z) \in \mathbb{R}^3\) можно записать как:
	\[
	\vec{a} = x \vec{i} + y \vec{j} + z \vec{k}.
	\]
\end{tbox}

\begin{tbox}{Производная по направлению}
	Пусть в \(\mathbb{R}^k\) задана функция \(y = f(\vec{x})\), где \(\vec{x} \in E \subset \mathbb{R}^k\) и \(\vec{x}_0 \in E \subset \mathbb{R}^k\). Выберем единичный вектор \(\vec{l} \in \mathbb{R}^k\), задающий направление, причем \(\|\vec{l}\| = 1\).

	Если существует предел:
	\[
	\exists \lim_{t \to 0} \frac{f(\vec{x}_0 + t \cdot \vec{l}) - f(\vec{x}_0)}{t},
	\]
	то его называют \textbf{производной функции \(f(\vec{x})\) в точке \(\vec{x}_0\) по направлению \(\vec{l}\)} и обозначают:
	\begin{equation*}
		\frac{\partial f(\vec{x}_0)}{\partial l} = \lim_{t \to 0} \frac{f(\vec{x}_0 + t \cdot \vec{l}) - f(\vec{x}_0)}{t}.
	\end{equation*}
\end{tbox}

\begin{tbox}{Частная производная}
	Частная производная функции \(f(\vec{x})\) по переменной \(x_i\) определяется как:
	\begin{equation*}
		\frac{\partial f(\vec{x}_0)}{\partial x_i} = \lim_{\Delta x_i \to 0} \frac{\Delta_i f(\vec{x}_0)}{\Delta x_i},
	\end{equation*}
	где
	\[
	\Delta_i f(\vec{x}_0) = f(x_{0_1}, ..., x_{0_{i-1}}, x_{0_i} + \Delta x_i, x_{0_{i+1}}, ..., x_{0_k}) - f(x_{0_1}, ..., x_{0_{i-1}}, x_{0_i}, x_{0_{i+1}}, ..., x_{0_k}).
	\]

	Разложение приращения аргумента по базису:
	\[
	\Delta \vec{x} = (0, \dots, 0, \Delta x_i, 0, \dots, 0) = \Delta x_i \cdot \vec{e}_i.
	\]

	Тогда:
	\begin{equation*}
		\frac{\partial f(\vec{x}_0)}{\partial x_i} = \lim_{\Delta x_i \to 0} \frac{f(\vec{x}_0 + \Delta x_i \cdot \vec{e}_i) - f(\vec{x}_0)}{\Delta x_i}.
	\end{equation*}

	Если обозначить \(\Delta x_i = t\), то:
	\begin{equation*}
		\frac{\partial f(\vec{x}_0)}{\partial x_i} = \lim_{t \to 0} \frac{f(\vec{x}_0 + t \cdot \vec{e}_i) - f(\vec{x}_0)}{t} = \frac{\partial f(\vec{x}_0)}{\partial e_i}.
	\end{equation*}

	Таким образом, \textbf{частная производная по переменной \(x_i\)} — это \textbf{производная по направлению соответствующего базисного вектора \(\vec{e}_i\)}.
\end{tbox}

\begin{tbox}{Производная по направлению для функции двух переменных}
	Рассмотрим функцию \(z = f(x, y)\) и точку \(M_0(x_0, y_0)\):

	\begin{align*}
		\frac{\partial z(M_0)}{\partial x}  = \lim_{\Delta x \to 0} \frac{f(x_0 + \Delta x, y_0) - f(x_0, y_0)}{\Delta x} = \frac{\partial f(M_0)}{\partial i} \quad& \text{(\Cref{fig:2.11.1.1})}\\
		\frac{\partial z(M_0)}{\partial y}  = \lim_{\Delta y \to 0} \frac{f(x_0, y_0 + \Delta y) - f(x_0, y_0)}{\Delta y} = \frac{\partial f(M_0)}{\partial i} \quad& \text{(\Cref{fig:2.11.1.2})}\\
		\frac{\partial z(M_0)}{\partial l} = \lim_{\Delta t \to 0} \frac{f(x_0 + t \cos \alpha, y_0 + t \cos \beta) - f(x_0, y_0)}{t} \quad&\text{(\Cref{fig:2.11.1.3})}
	\end{align*}

	Пусть \(\vec{l} \in \mathbb{R}^2\), \(\|\vec{l}\| = 1\), \(\vec{l} = (\cos \alpha, \cos \beta)\), где:
	\begin{itemize}
		\item \(\alpha\) — угол между вектором \(\vec{l}\) и осью \(Ox\),
		\item \(\beta\) — угол между вектором \(\vec{l}\) и осью \(Oy\).
	\end{itemize}
	\begin{figure}[H]
		\centering
		\begin{minipage}{0.32\linewidth}
			\centering
			\includegraphics[width=\linewidth]{2.pdf}
			\subcaption{ }
			\label{fig:2.11.1.1}
		\end{minipage}
		\begin{minipage}{0.32\linewidth}
			\centering
			\includegraphics[width=\linewidth]{3.pdf}
			\subcaption{ }
			\label{fig:2.11.1.2}
		\end{minipage}
		\begin{minipage}{0.32\linewidth}
			\centering
			\includegraphics[width=\linewidth]{4.pdf}
			\subcaption{ }
			\label{fig:2.11.1.3}
		\end{minipage}
		\caption{ }
		\label{fig:2.11.1}
	\end{figure}
\end{tbox}

\begin{tbox}{Теорема о связи производной по направлению с градиентом}
	Если функция $y = f(\vec{x})$ дифференцируема в точке $\vec{x}_0 \in E \subset \mathbb{R}^k$, то существуем производная по любому направлению $\vec{l}$, $\vec{l} \in \mathbb{R}^k$, $||\vec{l} = 1||$, и вычисляется это производная по формуле:
	\begin{equation*}
		\frac{\partial f(\vec{x}_0)}{\partial l} = (\grad f(\vec{x}_0), \, l)
	\end{equation*}
	\subsubsection*{Доказательство}

	Так как функция $y = f(\vec{x})$ дифференцируема в точке $\vec{x}_0$, то ее приращения в этой точке можно представить в виде:
	\begin{equation} \label{eq:6.1}
		\Delta f(\vec{x}_0) = f(\vec{x}) - f(\vec{x}_0) = \left(\vec{A}, \, \Delta \vec{x}\right) + \left(\vec{\alpha}(\Delta \vec{x}), \, \Delta \vec{x}\right),
	\end{equation}
	где $\vec{A} = \grad f(\vec{x}_0)$ -- постоянный вектор, координатами этого вектора является частные производные.
	\begin{equation*}
		\begin{aligned}
			\vec{A} = \left(\frac{\partial f(\vec{x}_0)}{\partial x_1}, \, \frac{\partial f(\vec{x}_0)}{\partial x_2}, ..., \frac{\partial f(\vec{x}_0)}{\partial x_k}\right) && \vec{\alpha}(\Delta \vec{x}) \to \vec{\theta} && \Delta \vec{x} \to \vec{\theta}
		\end{aligned}
	\end{equation*}
	Точку представим в виде:
	\begin{align*}
		\vec{x} = \Delta \vec{x} + \vec{x}_0 = \left[\Delta \vec{x} = t \cdot \vec{l}\,\right] = \vec{x}_0 + t \cdot \vec{l} && \Delta \vec{x} \to \vec{0} \text{, при }t \to 0
	\end{align*}

	Подставим в выражение \cref{eq:6.1}:
	\begin{multline*}
		f(\vec{x}_0 + t \cdot \vec{l}) - f(\vec{x}_0) = \left(\grad f(\vec{x}_0), \, t \cdot \vec{l}\,\right) + \left(\vec{\alpha}(t \cdot \vec{l}), \, t \cdot \vec{l}\,\right) = \\= \left(\grad f(\vec{x}_0), \, \vec{l} \,\right) \cdot t + \left(\vec{\alpha}(t \cdot \vec{l}), \, \vec{l}\,\right) \cdot t
	\end{multline*}
	\[\lim_{t \to 0} \frac{f(\vec{x}_0 + t \cdot \vec{l}) - f(\vec{x}_0)}{t} = \lim_{t \to 0} \left[\left(\grad f(\vec{x}_0), \vec{l}\,\right) + \left(\vec{\alpha}(t \cdot \vec{l}), \vec{l}\,\right)\right] = \left(\grad d \, f(\vec{x}_0), \vec{l}\,\right)\]
\end{tbox}

\subsubsection*{Пример}
Дана функция:
\[ u = xy - z^2 \]

Требуется найти производную этой функции в направлении биссектрисы первого координатного угла \( xoy \) в точке \( M_0(-9, 12, 10) \).

\vspace{0.5em}
\begin{enumerate}
	\item \textbf{Вычисление градиента функции \( u \):}
	\[
	\grad u = \left( \frac{\partial u}{\partial x}, \frac{\partial u}{\partial y}, \frac{\partial u}{\partial z} \right) = y \cdot \vec{i} + x \cdot \vec{j} - 2z \cdot \vec{k}
	\]
	В точке \( M_0 \):
	\[
	\grad u(M_0) = 12 \vec{i} - 9 \vec{j} - 20 \vec{k}
	\]
	\item \textbf{Направление биссектрисы \( xoy \):}\\
	Биссектриса первого координатного угла имеет направляющие косинусы:
	\[
	\cos \alpha = \frac{1}{\sqrt{2}}, \quad \cos \beta = \frac{1}{\sqrt{2}}, \quad \cos \gamma = 0
	\]
	Таким образом, вектор направления:
	\[
	\vec{l} = \left(\cos \alpha, \cos \beta, \cos \gamma\right) = \frac{1}{\sqrt{2}} \vec{i} + \frac{1}{\sqrt{2}} \vec{j}
	\]
	\item \textbf{Производная в направлении \( \vec{l} \):}
	\[
	\frac{\partial u(M_0)}{\partial l} = \left( \grad u(M_0), \vec{l} \,\right) = 12 \cdot \frac{1}{\sqrt{2}} + (-9) \cdot \frac{1}{\sqrt{2}} = \frac{3}{\sqrt{2}}
	\]

	\textbf{Итоговый ответ:}
	\[
	\frac{\partial u(M_0)}{\partial l} = \frac{3}{\sqrt{2}}
	\]
\end{enumerate}
%\begin{figure}[H]
%	\centering
%	\includegraphics[width=0.5\linewidth]{5.pdf} % Укажите высоту, если нужно
%	\caption{ }
%	\label{fig:18}
%\end{figure}
	\subsection{Условия монотонности функции в заданном направлении.}

\textbf{Определения: } Функция $y = f(\vec{x})$, $\vec{x} \in E \subset \mathbb{R}^k$, называется монотонно возрастающей в точке $\vec{x}_0 \in E \subset \mathbb{R}^k$ в направлении вектора $\vec{l} \subset \mathbb{R}^k$, $||\vec{l}|| = 1$, если существует окрестность точки $x_0$, такая, что для любых точке $\vec{x}_1$ и $\vec{x}_2$, $\forall \vec{x}_1$ и $\forall \vec{x}_2$ из этой окрестности, лежащиз на отрезке коллинеарным с вектором $\vec{l}$, так что $\vec{x}_1$ предшествует $\vec{x}_0$, а $\vec{x}_2$ следует за $\vec{x}_0$, выполняется неравенство.
\begin{equation} \label{eq:6.2}
	f(\vec{x}_1) < f(\vec{x}_0) \quad \text{ и } \quad f(\vec{x}_0) < f(\vec{x}_2)
\end{equation}

Если знак неравенства поменять на противоположный, то:
\begin{equation} \label{eq:6.3}
	f(\vec{x}_1) > f(\vec{x}_0) \quad \text{ и } \quad f(\vec{x}_0) > f(\vec{x}_2)
\end{equation}
получим определение функции монотонно убывающей в точке $\vec{x}_0$ в направлении $\vec{l}$.

\begin{tbox}{Теорема (Без доказательства)}
	Достаточное условие монотонности функции в заданном направлении.

	Пусть $y = f(\vec{x})$, дифференцируема в точке $\vec{x}_0 \in E \subset \mathbb{R}^k$,  $||\vec{l}|| = 1$, $\vec{l} \subset \mathbb{R}^k$ задает направление.

	\begin{itemize}
		\item Если $\displaystyle \frac{\partial \, F(\vec{x}_0)}{\partial l} > 0$, то функция $f(\vec{x})$ является монотонно возрастающей в точке $\vec{x}_0$ в направлении вектора $\vec{l}$.
		\item Если $\displaystyle\frac{\partial f(\vec{x}_0)}{\partial l} < 0$, то функция $f(\vec{x})$ является монотонно убывающей в точке $\vec{x}_0$ в направлении вектора $\vec{l}$.
		\item Если $\displaystyle\frac{\partial f(\vec{x}_0)}{\partial l} = 0$, то функция $f(\vec{x})$ не изменяется в направлении вектора $\vec{l}$, то есть равна $const$.
	\end{itemize}
\end{tbox}
	\subsection{Геометрический смысл градиента}
Пусть $y = f(\vec{x})$, производная по направлению $\vec{l}$ ($\vec{l} \in \mathbb{R}^k$, $||\vec{l}|| = 1$) в точке $\vec{x}_0$ вычисляется по формуле:
\begin{align*}
	\frac{\partial f(\vec{x}_0)}{\partial l} = \left(\grad \, f(\vec{x}_0), \, \vec{l}\right) = ||\grad \, f(\vec{x}_0)|| \cdot ||\vec{l}|| \cdot \cos \omega && \omega = \widehat{\left(\grad \, f(\vec{x}_0), \vec{l}\right)}
\end{align*}
\begin{equation*}
	\boxed{\frac{\partial f(\vec{x}_0)}{\partial l} = ||\grad \, f(\vec{x}_0)|| \cdot \cos \omega}
\end{equation*}

\begin{enumerate}
	\item Пусть $\grad \, f(\vec{x}_0) \uparrow \uparrow \vec{l}$, то $\omega = 0$, $\cos \omega = 1$.
	\begin{equation*}
		\frac{\partial \, f(\vec{x}_0)}{\partial l} = ||\grad \, f(\vec{x}_0)|| > 0.
	\end{equation*}

	В направлении вектора $\grad \, f(\vec{x}_0)$ -- функция монотонно возрастает, при этом производная по направлению принимает \textbf{максимальное} значение, это означает, что в направлении вектора $\grad \, f(\vec{x}_0)$ -- функция быстрее всего возрастает из точки $x_0$, поэтому вектор $\grad \, f(\vec{x}_0)$ определяет наибыстрейший подъема функции из точки $x_0$.

	\item Пусть $\grad \, f(\vec{x}_0) \uparrow \downarrow \vec{l}$, то $\omega = \pi$, $\cos \omega = -1$.
	\begin{equation*}
		\frac{\partial \, f(\vec{x}_0)}{\partial l} = -||\grad \, f(\vec{x}_0)|| < 0.
	\end{equation*}

	Тогда по теореме в направлении противоположному вектору $\grad \, f(\vec{x}_0)$ функция будет монотонно убывать, причем в направлении наибыстрейшего спуска функции из точки $x_0$, так как:
	\begin{equation*}
		\left|\frac{\partial f(\vec{x}_0)}{\partial l}\right| = \left|\grad \, f(\vec{x}_0)\right|
	\end{equation*}
	-- принимает модуль максимальное значение.
	\item Пусть $\grad \, f(\vec{x}_0) \perp \vec{l}$, то $\omega = \frac{\pi}{2}$, $\cos \omega = 0$.
	\begin{equation*}
		\frac{\partial \, f(\vec{x}_0)}{\partial l} = 0.
	\end{equation*}
	То есть по направлению перпендикулярном градиенту $\grad \, f(\vec{x}_0)$, получается что функция $f(\vec{x})$ в точке $\vec{x}_0$ функция не изменяется.
\end{enumerate}

	\subsection{Экстремум функции многих переменных}

\begin{tbox}{Определение 1}
	Функция $y = f(\vec{x})$ имеет в точке $\vec{x}_0 \subset \mathbb{R}^k$ локальный максимум, если существует окрестность точки $\vec{x}_0$, такая, что для $\forall \vec{x}$ из этой окрестности выполняется неравенство:
	\begin{equation*}
		f(\vec{x}) < f(\vec{x}_0)
	\end{equation*}
	\begin{equation}\label{eq:6.4}
		(\exists \delta > 0)(\forall \vec{x} \in E, \, 0 < ||\vec{x} - \vec{x}_0|| < \delta): \, f(\vec{x}) < f(\vec{x}_0)
	\end{equation}
\end{tbox}

Если в определении 1 неравенство нестрогое: $f(\vec{x}) <= f(\vec{x}_0)$ -- то максимум называется \textbf{нестрогим}, аналогично $f(\vec{x})$ -- имеет в точке $\vec{x}_0$ -- локальный минимум, если:
\begin{equation} \label{eq:6.5}
	(\exists \delta > 0)(\forall \vec{x} \in E, \, 0 < ||\vec{x} - \vec{x}_0|| < \delta): \, f(\vec{x}) > f(\vec{x}_0)
\end{equation}

Если неравенство \cref{eq:6.5} нестрогое $f(\vec{x}) >= f(\vec{x}_0)$ -- то минимум называется нестрогим.

\begin{tbox}{Теорема 1. Необходимое условие экстремума}
	Пусть функция \( y = f(x) \) дифференцируема в точке \( \vec{x}_0 \in \mathbb{E} \subset \mathbb{R}^k \) и имеет в этой точке локальный экстремум (максимум или минимум).

	Тогда все частные производные 1-го порядка равны нулю в этой точке:
	\begin{equation*}
		\frac{\partial f(x_0)}{\partial x_1} = 0, \quad \frac{\partial f(x_0)}{\partial x_2} = 0, \quad \dots \quad \frac{\partial f(x_0)}{\partial x_k} = 0.
	\end{equation*}

	\textbf{Доказательство: } Пусть $\vec{x}_0 = (x_{0_1}, x_{0_2}, \cdots, x_{0_k})$,
\end{tbox}

Рассмотрим функцию \( y = f(x_1, x_2, \dots, x_k) \). Зафиксируем переменные \( x_2, x_3, \dots, x_k \), полагая их равными:
\begin{equation}
	x_2 = x_0, \quad x_3 = x_0, \quad \dots, \quad x_k = x_0.
\end{equation}

Тогда получаем функцию \( g \), которая зависит от одной переменной \( x_1 \):
\begin{equation}
	y = f(x_1, x_0, \dots, x_0) = g(x_1).
\end{equation}

\begin{tbox}{Необходимое условие для функции одной переменной}
	Так как точка \( x_0 \) является точкой экстремума для функции \( y = f(x_1, x_2, \dots, x_k) \), то она также является точкой экстремума для функции \( g(x_1) \). Согласно необходимому условию экстремума:
	\begin{equation}
		\frac{dg(x_0)}{dx_1} = 0 \Rightarrow \frac{\partial f(x_0)}{\partial x_1} = 0.
	\end{equation}
\end{tbox}

Аналогично, фиксируя другие переменные, получаем:
\begin{equation}
	\frac{dg(x_0)}{dx_2} = 0 \Rightarrow \frac{\partial f(x_0)}{\partial x_2} = 0.
\end{equation}

Таким образом, все частные производные 1-го порядка равны нулю в точке экстремума.

	\subsection{Понятие квадратичной формы и критерии Сильвестра}

\begin{tbox}{Квадратичная форма}
	Рассмотрим симметрическую матрицу размером $k \times k$:
	\begin{equation*}
		A =
		\begin{pmatrix}
			a_{11} & a_{12} & a_{13} & \cdots & a_{1k} \\
			a_{21} & a_{22} & a_{23} & \cdots & a_{2k} \\
			a_{31} & a_{32} & a_{33} & \cdots & a_{3k} \\
			\vdots & \vdots & \vdots & \ddots & \vdots \\
			a_{k1} & a_{k2} & a_{k3} & \dots & a_{kk} \\
		\end{pmatrix}
		\quad a_{ij} = a_{ji}
	\end{equation*}

	Главными минорами матрицы $A$ называются следующие определители:
	\begin{align*}
		A_1 = a_{11} &&
		A_2 = \begin{vmatrix}
			a_{11} & a_{12} \\
			a_{21} & a_{22}
		\end{vmatrix} &&
		A_{3} = \begin{vmatrix}
			a_{11} & a_{12} & a_{13} \\
			a_{21} & a_{22} & a_{23} \\
			a_{31} & a_{32} & a_{33}
		\end{vmatrix} &&
		A_k = \begin{vmatrix}
			a_{11} & a_{12} & a_{13} & \cdots & a_{1k} \\
			a_{21} & a_{22} & a_{23} & \cdots & a_{2k} \\
			a_{31} & a_{32} & a_{33} & \cdots & a_{3k} \\
			\vdots & \vdots & \vdots & \ddots & \vdots \\
			a_{k1} & a_{k2} & a_{k3} & \dots  & a_{kk}
		\end{vmatrix}
	\end{align*}

	Пусть $x_1$, $x_2$, $x_3$, $\dots$, $x_k$ -- некоторые переменные величины, из них и элементов матрицы $A$ составим выражения:
	\begin{align*}
		P(x_1, x_2, \dots, x_k) = \,
		& a_{11} \cdot x_{1}^2 + a_{12} \cdot x_{1} x_{2} + a_{13} \cdot x_{1} x_{3} + ... + a_{1k} \cdot x_{1} x_{k} + \\
		&a_{21} \cdot x_{2} x_{1} + a_{22} \cdot x_{2}^2 + a_{23} \cdot x_{2} x_{3} + ... + a_{2k} \cdot x_{2} x_{k} + \\
		&a_{31} \cdot x_{3} x_{1} + a_{32} \cdot x_{3} x_{2} + a_{33} \cdot x_{3}^2 + ... + a_{3k} \cdot x_{3} x_{k} + \\
		&\dots\\
		&a_{k1} \cdot x_{k} x_{1} + a_{k2} \cdot x_{k} x_{2} + a_{k3} \cdot x_{k} x_{3} + ... + a_{kk} \cdot x_{k}^2
	\end{align*}

	Так как матрица $A$ симметрична ($a_{ij} = a_{ji}$), тогда:
	\begin{equation} \label{eq:7.1}
		\begin{aligned}
			P(x_1, x_2, \dots, x_k) = &(a_{11} \cdot x_{1}^2 + a_{22} \cdot x_{2}^2 + a_{33} \cdot x_{3}^2 + ... + a_{kk} \cdot x_k^2) + \\
			&(2 a_{12} \cdot x_1 x_2 + 2 a_{13} \cdot x_1 x_3 + 2 a_{23} \cdot x_2 x_3 + ... + 2 a_{k - 1, k} \cdot x_{k-1} x_{k})
		\end{aligned}
	\end{equation}

	Выражение, включающее в себя квадраты всех переменных и всевозможные их попарные произведения, называется \textbf{квадратичной формой}.
	\begin{align*}
		\text{Пусть $k = 2$} && P(x_1, x_2) = a_{11} \cdot x_1^2 + a_{22} \cdot x_2^2 + 2 a_{12} \cdot x_1 x_2,\\
		\text{Пусть $k = 3$} && P(x_1, x_2, x_3) = a_{11} \cdot x_1^2 + a_{22} \cdot x_2^2 + a_{33} \cdot x_3^2 + \\&&+ 2 a_{12} \cdot x_1 x_2 + 2 a_{13} \cdot x_1 x_3 + 2 a_{23} \cdot x_2 x_3
	\end{align*}

	Если квадратичная форма \eqref{eq:7.1}, $P(x_1, x_2, ..., x_k) > 0$, при различных значения переменных, то ее называют \textbf{положительно определенной квадратичной формой}.

	Если $P(x_1, x_2, ..., x_k) < 0$, при всех значениях переменных $(x_1, x_2, ..., x_k)$, то ее называют \textbf{отрицательно определенной квадратичной формой}.

	Если $P(x_1, x_2, ..., x_k)$ принимает как положительное, так и отрицательное значение, при разных значениях переменных, то ее называют \textbf{знакопеременной квадратичной формой}.
\end{tbox}

\begin{tbox}{Критерии Сильвестра}
	Чтобы квадратичная форма матрицы $A$, являлась \textit{положительной определенной}, необходимо и достаточно, чтобы все главные миноры матрицы $A$ были положительны.

	\begin{equation*}
		P(x_1, x_2, ... x_k) > 0  \Leftrightarrow A_1 > 0, \, A_2 > 0, \, A_3 > 0, \, ...,A_k>0
	\end{equation*}

	Чтобы квадратичная форма матрицы $A$, являлась \textit{отрицательно определенной}, необходимо и достаточно, чтобы все знаки главных миноров матрицы $A$ чередовались, при чем $A_1 < 0$.

	\begin{equation*}
		P(x_1, x_2, ..., x_k) < 0 \Leftrightarrow A_1 < 0, \, A_2 > 0, \, A_3 < 0, ..., \, A_k > 0
	\end{equation*}
\end{tbox}

\begin{tbox}{Второй дифференциал функции многих переменных}
	Рассмотрим второй дифференциал для функции $k$ - независимых переменных.
	\begin{multline*}
		d^2 y = \left(\frac{\partial}{\partial x_1} dx_1 + \frac{\partial}{\partial x_2} dx_2 + \cdots + \frac{\partial}{\partial x_k} dx_k\right)^2 f(x_1, \ldots, x_k) \\
		= \left(\frac{\partial^2}{\partial x_1^2} dx_1^2 + \frac{\partial^2}{\partial x_2^2} dx_2^2 + \cdots + \frac{\partial^2}{\partial x_k^2} dx_k^2 \right. \\
		\left. + 2\frac{\partial^2}{\partial x_1 \partial x_2} dx_1 dx_2 + \cdots + 2\frac{\partial^2}{\partial x_{k-1} \partial x_k} dx_{k-1} dx_k\right) f \\
		= \frac{\partial^2 f}{\partial x_1^2} dx_1^2 + \frac{\partial^2 f}{\partial x_2^2} dx_2^2 + \cdots + \frac{\partial^2 f}{\partial x_k^2} dx_k^2 \\
		+ 2\frac{\partial^2 f}{\partial x_1 \partial x_2} dx_1 dx_2 + \cdots + 2\frac{\partial^2 f}{\partial x_{k-1} \partial x_k} dx_{k-1} dx_k
	\end{multline*}
	Если $x_0$ - стационарная точка.
	\begin{multline*}
		d^2 y\left(\vec{x}_0\right)= \,\frac{\partial^2 f\left(\vec{x}_0\right)}{\partial x_1^2} d x_1^2+\frac{\partial^2 f\left(\vec{x}_0\right)}{\partial x_2^2} \cdot d x_2^2+...+ \frac{\partial^2 f\left(\vec{x}_0\right)}{\partial x_k^2} \cdot d x_k^2+ \\
		+2\frac{\partial^2 f}{\partial x_1 \partial x_2} d x_1 d x_2+2\frac{\partial^2 f}{\partial x_1 \partial x_3} d x_1 d x_3+\ldots+2\frac{\partial^2 f\left(\vec{x}_0\right)}{\partial x_{k-1} \partial x_k} d x_{k-1} d x_k
	\end{multline*}

	$d^2 y$ в точке $x_0$ -- является квадратичной формой относительно дифференциалов независимых переменных $d x_1$, $d x_2$, ..., $d x_k$. Матрица этой квадратичной формы имеет вид:
	\begin{equation*}
		\begin{aligned}
			A = \begin{pmatrix}
				\frac{\partial^2 f}{\partial x_1^2} & \frac{\partial^2 f}{\partial x_1 \partial x_2} & \frac{\partial^2 f}{\partial x_1 \partial x_3} & \cdots & \frac{\partial^2 f}{\partial x_1 \partial x_k} \\
				\frac{\partial^2 f}{\partial x_2 \partial x_1} & \frac{\partial^2 f}{\partial x_2^2} & \frac{\partial^2 f}{\partial x_2 \partial x_3} & \cdots & \frac{\partial^2 f}{\partial x_2 \partial x_k} \\
				\vdots & \vdots & \vdots & \ddots & \vdots \\
				\frac{\partial^2 f}{\partial x_k \partial x_1} & \frac{\partial^2 f}{\partial x_k \partial x_2} & \frac{\partial^2 f}{\partial x_k \partial x_3} & \cdots & \frac{\partial^2 f}{\partial x_k \partial x_k} \\
			\end{pmatrix}
		\end{aligned}
	\end{equation*}

	Тогда для определения знака $d^2 y (\vec{x}_0)$ -- применяем критерий Сильвестра. Для этого находим в точке $\vec{x}_0$ -- главные миноры матрицы $A$ и определяет их знак, соответственно делаем вывод о знаке квадратичной формы.
\end{tbox}

\begin{tbox}{Теорема о достаточном условии вывода экстремума для функции двух переменных}
	Пусть функция $z = f(x,y)$ -- дифференцируемая в окрестностях стационарной точки $M_0(x_0, y_0)$ и имеет в этой точке непрерывную частную производную второго порядка. Тогда:
	\begin{enumerate}
		\item Если $f_{xx}''(M_0) \cdot f_{yy}''(M_0) - {f_{xy}''}^2(M_0) >0$ и $f_{xx}''(M_0) > 0$, то $M_0$ -- точка минимума.
		\item Если $f_{xx}''(M_0) \cdot f_{yy}''(M_0) - {f_{xy}''}^2(M_0) >0$ и $f_{xx}''(M_0) < 0$, то $M_0$ -- точка максимума.
		\item Если $f_{xx}''(M_0) \cdot f_{yy}''(M_0) - {f_{xy}''}^2(M_0) < 0$, то в точке $M_0$ экстремума нет.
	\end{enumerate}

	\subsubsection*{Доказательство}
	Следует из критерия Сильвестра, для этого запишем второй дифференциал функции $z = f(x,y)$ в стационарной точке $M_0$.
	\begin{equation*}
		\begin{aligned}
			d^2 z (M_0) = f_{xx}''(M_0) \,dx^2 + 2 f_{xy}''(M_0) \,dx\, dy + f_{yy}''(M_0) \, dy^2
		\end{aligned}
	\end{equation*}
	$d^2 z(M_0)$ -- является квадратичной формой относительно дифференциалов $dx$ и $dy$, а матрица этой квадратичной формы имеет вид.
	\begin{equation*}
		\begin{aligned}
			A = \begin{pmatrix}
				f_{xx}''(M_0) & f_{xy}''(M_0) \\
				f_{yx}''(M_0) & f_{yy}''(M_0)
			\end{pmatrix}
		\end{aligned}
	\end{equation*}

	Запишем главные миноры матрицы $A$:
	\begin{equation*}
		\begin{aligned}
			A_1 = f_{xx}''(M_0) && A_2 = \begin{vmatrix}
				f_{xx}''(M_0) & f_{xy}''(M_0) \\
				f_{yx}''(M_0) & f_{yy}''(M_0)
			\end{vmatrix} = f_{xx}''(M_0) \cdot f_{yy}''(M_0) - {f_{xy}''}^2(M_0)
		\end{aligned}
	\end{equation*}

	Тогда по критерию Сильвестра:
	\begin{enumerate}
		\item если $A_1 > 0$ и $A_2 > 0$, то квадратичная форма является \textit{положительно определенной}, т.е. $d^2 z(M_0) > 0$ и по теореме 2 точка $M_0$ является точкой \textit{минимума}.
		\item если $A_1 < 0$ и $A_2 > 0$, то квадратичная форма является \textit{отрицательно определенной}, т.е. $d^2 z(M_0) < 0$ и по теореме 2 точка $M_0$ является точкой \textit{максимума}.
		\item если $A_2 = f_{xx}''(M_0) \cdot f_{yy}''(M_0) - {f_{xy}''}^2(M_0) < 0$, в этом случае квадратичная форма является знакопеременной квадратичной формой, т.к. оно не будет ни положительной, ни отрицательной.
	\end{enumerate}

	Если $d^2 z (M_0)$ -- не определен по знаку, то поп теореме 3 в точке $M_0$ экстремума нет. \\

	\textbf{Замечания: } В случае, если $A_2 = f_{xx}''(M_0) \cdot f_{yy}''(M_0) - {f_{xy}''}^2(M_0) = 0$, то наличия или отсутствие экстремума необходимо доказать по определению, этот случай соответствует тому, что $d^2 z(M_0) = 0$.
\end{tbox}
	\subsection{Условный экстремум функции многих переменных}
\begin{tbox}{Условный экстремум}
	Экстремум функции, на аргументы которой наложены дополнительные условия связи, называется \textbf{условным экстремумом}.
\end{tbox}

\begin{figure}[h]
	\centering
	\begin{minipage}[c]{0.4\linewidth} % [c] для выравнивания по центру
		\centering
		\includegraphics[width=\linewidth]{7.pdf} % Укажите высоту, если нужно
		\caption{Графическая иллюстрация условного экстремума}
		\label{fig:17}
	\end{minipage}
\end{figure}

\subsubsection*{Пример условного экстремума (Рис. \ref{fig:17})}
Пусть $z = x^2 + y^2$ при условии связи $x + y = 1$. Точка $M_0(0, 0)$ является точкой минимума (абсолютный экстремум).

Экстремум на линии пересечения плоскости $x + y = 1$ и параболоида $z = x^2 + y^2$ является условным экстремумом. Для его нахождения выразим переменную $y$ из условия связи и подставим в уравнение функции.

Получаем функцию одной независимой переменной, которую исследуем на экстремум:
\begin{gather*}
	x + y = 1 \Rightarrow y = 1 - x, \\
	z = x^2 + y^2 = x^2 + (1 - x)^2 = x^2 + 1 - 2x + x^2 = 2x^2 - 2x + 1.
\end{gather*}
\begin{align*}
	z_x' &= 4x - 2 = 0 \Rightarrow x = \frac{1}{2} \text{ — стационарная точка}, \\
	z_{xx}'' &= 4 > 0 \Rightarrow x = \frac{1}{2} \text{ — точка минимума функции } z = 2x^2 - 2x + 1, \\
	y &= 1 - x = 1 - \frac{1}{2} = \frac{1}{2}; \quad \left(\frac{1}{2}; \frac{1}{2}\right) \text{ — точка условного минимума}, \\
	z_{\text{min}} &= z\left(\frac{1}{2}; \frac{1}{2}\right) = \left(\frac{1}{2}\right)^2 + \left(\frac{1}{2}\right)^2 = \frac{1}{2}.
\end{align*}

Если уравнений связи несколько или они нелинейные, то свести задачу условного экстремума к исследованию функций одной или нескольких переменных бывает сложно. В этом случае применяют \textit{метод неопределённых множителей Лагранжа}.

\begin{tbox}{Метод множителей Лагранжа}
	В общем случае задача ставится следующим образом:

	Пусть задана функция $y = f(x_1, x_2, \dots, x_k)$, а независимые переменные $x_1, x_2, \dots, x_k$ связаны условиями, число которых меньше числа независимых переменных:
	\begin{equation} \label{eq:8.1}
		\begin{cases}
			\varphi_1(x_1, x_2, \dots, x_k) = 0, \\
			\varphi_2(x_1, x_2, \dots, x_k) = 0, \\
			\dots, \\
			\varphi_n(x_1, x_2, \dots, x_k) = 0,
		\end{cases}
		\quad n < k.
	\end{equation}

	Составим функцию Лагранжа:
	\begin{multline} \label{eq:8.2}
		L = f(x_1, x_2, \dots, x_k) + \lambda_1 \varphi_1(x_1, x_2, \dots, x_k) + \lambda_2 \varphi_2(x_1, x_2, \dots, x_k) + \dots \\ \dots + \lambda_n \varphi_n(x_1, x_2, \dots, x_k).
	\end{multline}

	Здесь $\lambda_1, \lambda_2, \dots, \lambda_n$ — неопределённые множители Лагранжа, которые находятся из условий:
	\begin{equation*}
		\begin{cases}
			\frac{\partial L}{\partial \lambda_1} = \varphi_1(x_1, x_2, \dots, x_k) = 0, \\
			\frac{\partial L}{\partial \lambda_2} = \varphi_2(x_1, x_2, \dots, x_k) = 0, \\
			\dots, \\
			\frac{\partial L}{\partial \lambda_n} = \varphi_n(x_1, x_2, \dots, x_k) = 0.
		\end{cases}
	\end{equation*}

	Задача поиска условного экстремума сводится к поиску обычного экстремума для функции Лагранжа \eqref{eq:8.2}. Необходимое условие экстремума:
	\begin{equation} \label{eq:8.3}
		\begin{cases}
			\frac{\partial L}{\partial x_1} = 0, \\
			\frac{\partial L}{\partial x_2} = 0, \\
			\dots, \\
			\frac{\partial L}{\partial x_k} = 0, \\
			\frac{\partial L}{\partial \lambda_1} = \varphi_1(x_1, x_2, \dots, x_k) = 0, \\
			\frac{\partial L}{\partial \lambda_2} = \varphi_2(x_1, x_2, \dots, x_k) = 0, \\
			\dots, \\
			\frac{\partial L}{\partial \lambda_n} = \varphi_n(x_1, x_2, \dots, x_k) = 0.
		\end{cases}
	\end{equation}

	Получаем систему $k + n$ уравнений относительно неизвестных $x_1, x_2, \dots, x_k$ и $\lambda_1, \lambda_2, \dots, \lambda_n$.

	Решая систему \eqref{eq:8.3}, находим стационарные точки $M_0(x_{0_1}, x_{0_2}, \dots, x_{0_k})$ и значения множителей Лагранжа $\lambda_1 = \lambda_{0_1}, \lambda_2 = \lambda_{0_2}, \dots, \lambda_n = \lambda_{0_n}$.
\end{tbox}

\begin{tbox}{Второй дифференциал функции Лагранжа}
	Дальнейшее исследование стационарных точек на экстремум связано с анализом знака второго дифференциала с учётом дифференциальных условий связи.

	Второй дифференциал функции Лагранжа в точке $M_0$:
	\begin{equation*}
		\begin{aligned}
			d^2 L(M_0) = &\frac{\partial^2 L(M_0)}{\partial x_1^2} dx_1^2 + \frac{\partial^2 L(M_0)}{\partial x_2^2} dx_2^2 + \dots + \frac{\partial^2 L(M_0)}{\partial x_k^2} dx_k^2 \, + \\
			&+ 2 \frac{\partial^2 L(M_0)}{\partial x_1 \partial x_2} dx_1 dx_2 + 2 \frac{\partial^2 L(M_0)}{\partial x_1 \partial x_3} dx_1 dx_3 + \dots + 2 \frac{\partial^2 L(M_0)}{\partial x_{k-1} \partial x_k} dx_{k-1} dx_k.
		\end{aligned}
	\end{equation*}

	Так как переменные $x_1, x_2, \dots, x_k$ связаны условиями \eqref{eq:8.1}, их дифференциалы также связаны. Найдём дифференциалы условий связи в точке $M_0$:
	\begin{gather*}
		\varphi_1(x_1, x_2, \dots, x_k) = 0 \quad \Big| \, d, \\
		d\varphi_1(M_0) = \frac{\partial \varphi_1(M_0)}{\partial x_1} dx_1 + \frac{\partial \varphi_1(M_0)}{\partial x_2} dx_2 + \dots + \frac{\partial \varphi_1(M_0)}{\partial x_k} dx_k = \sum_{i=1}^k \frac{\partial \varphi_1(M_0)}{\partial x_i} dx_i = 0, \\
		\varphi_2(x_1, x_2, \dots, x_k) = 0 \quad \Big| \, d, \\
		d\varphi_2(M_0) = \frac{\partial \varphi_2(M_0)}{\partial x_1} dx_1 + \frac{\partial \varphi_2(M_0)}{\partial x_2} dx_2 + \dots + \frac{\partial \varphi_2(M_0)}{\partial x_k} dx_k = \sum_{i=1}^k \frac{\partial \varphi_2(M_0)}{\partial x_i} dx_i = 0, \\
		\dots, \\
		\varphi_n(x_1, x_2, \dots, x_k) = 0 \quad \Big| \, d, \\
		d\varphi_n(M_0) = \frac{\partial \varphi_n(M_0)}{\partial x_1} dx_1 + \frac{\partial \varphi_n(M_0)}{\partial x_2} dx_2 + \dots + \frac{\partial \varphi_n(M_0)}{\partial x_k} dx_k = \sum_{i=1}^k \frac{\partial \varphi_n(M_0)}{\partial x_i} dx_i = 0.
	\end{gather*}

	Исследуем знак второго дифференциала с учётом этих условий.
\end{tbox}

\subsubsection*{Пример метода множителей Лагранжа}
Задана функция $z = x^2 + xy + y^2$ с условием $x + y = 2$, где $x$ и $y$ — независимые переменные. Введём функцию Лагранжа:
\[
L = x^2 + xy + y^2 + \lambda(x + y - 2).
\]

Частные производные:
\[
\begin{cases}
	2x + y + \lambda = 0, \\
	x + 2y + \lambda = 0, \\
	x + y = 2.
\end{cases}
\]

Решаем систему:
\[
y = 2 - x, \quad 2x + (2 - x) + \lambda = 0, \quad x + 2(2 - x) + \lambda = 0.
\]
\[
x = 1, \quad y = 1, \quad \lambda = -3.
\]

Вторые производные:
\[
\frac{\partial^2 L}{\partial x^2} = 2, \quad \frac{\partial^2 L}{\partial x \partial y} = 1, \quad \frac{\partial^2 L}{\partial y^2} = 2.
\]

Квадратичная форма:
\[
d^2 L = 2 dx^2 + 2 dx dy + 2 dy^2.
\]
При $dx + dy = 0$ (так как $x + y = 2$):
\[
d^2 L = 2 dx^2 > 0.
\]

Точка $(1, 1)$ — условный минимум. Значение:
\[
z(1, 1) = 3.
\]

\textbf{Ответ:} $z_{\text{min}} = 3$ в точке $(1, 1)$.
\end{document}